\section{The Offline Algorithm}

Let $\varphi$ be an STL formula and $(S,{\hb})$ be a distributed signal of $n$ signals $x_1, \ldots x_n$ over the temporal domain $[0,d)$.

\begin{enumerate}
	\item 
	Enumerate the subformulas of $\varphi$ such that each formula has an enumeration number smaller than the numbers of all its subformulas.
	Let $\varphi_1, \ldots, \varphi_m$ be such an enumeration.
	Note that $\varphi_1 = \varphi$.
	
	\item
	Compute the canonical segmentation $G_{S} = \{ [t_1, t_2), \ldots, [t_\ell, t_{\ell+1}) \}$ of $(S, {\hb})$.
	
	\item
	Compute the value expressions of signals with respect to the canonical segmentation, i.e., for each $1 \leq j \leq \ell$ and $1 \leq i \leq n$, compute $\gamma([t_j, t_{j+1}), i)$.
	
	\item
	Compute the value expressions of subformulas with respect to the canonical segmentation, i.e., for each $1 \leq j \leq \ell$ and $1 \leq k \leq m$, compute $[S, t_j \models \varphi_k]$
	%For $i = k .. 1$, for $j = m .. 1$, compute $[S, t_i \models \varphi_j]$.
	
	\item
	Output the set $\mathsf{Out}(\varphi, S, {\hb}, \varepsilon, \delta) = \destutter([S,t_1 \models \varphi_1] \cdot \ldots \cdot [S,t_\ell \models \varphi_1])$ of value expressions. %\TODO: match the semantics -- \alert{$\first([S',0 \models \varphi_1'])$}.
\end{enumerate}

\begin{claim}
	The algorithm is correct.
	For all STL formulas $\varphi$, and distributed signals $(S, {\hb})$ of bounded clock skew $\varepsilon$ and bounded variability $\delta$, we have $\mathsf{Goal}(\varphi, S, {\hb}, \varepsilon, \delta) \subseteq \mathsf{Out}(\varphi, S, {\hb}, \varepsilon, \delta)$.
\end{claim}
\begin{proof} \ege{To be checked.}
	Let $\varphi$ be an STL formula, $(S, {\hb})$ be a distributed signal of bounded clock skew $\varepsilon$ and bounded variability $\delta$ with $S = (x_1, \ldots, x_n)$.
	
	Recall that each element of $\mathsf{Goal}(\varphi, S, {\hb}, \varepsilon, \delta)$ is a value expression for the satisfaction signal of $\varphi$ for some trace of $(S,{\hb})$.
	Let $v \in \mathsf{Goal}(\varphi, S, {\hb}, \varepsilon, \delta)$ be arbitrary.
	To conclude that $v \in \mathsf{Out}(\varphi, S, {\hb}, \varepsilon, \delta)$, we argue that the semantics of our logic explores all admissible interleavings (subject to ${\hb}$, $\varepsilon$, and $\delta$) of the edges of $S$.
	
	First, notice that the canonical segmentation $G_S = \{ I_1, \ldots, I_\ell \}$ of $(S,{\hb})$ and the function $\gamma$ computing the corresponding value expressions for the signals capture all potential behaviors of the individual signals in $S$.
	In particular, for each trace $w = (x_1', \ldots, x_n')$ of $(S,{\hb})$ and for each $i$, it is easy to see that $\mu(x_i') \in \destutter(\gamma(I_1, i) \ldots \gamma(I_\ell, i))$.
	
	Now, we argue that the semantics of our logic preserve this property.
	For atomic propositions and conjunction, simply observe that the asynchronous product operation captures all the admissible interleavings.
	For untimed until, observe that we consider the asynchronous product and apply bitwise-until operations whose definitions coincide with the usual recursive definitions.
	For timed eventually and timed always, observe that the profiles of intervals capture all potential overlappings of the operation's interval and the canonical segments, to which we apply the corresponding bitwise operations which coincide with the usual definitions.
	The case of negation is trivial.
\end{proof}

\begin{claim}
	\TODO: The algorithm runs in ... %time and space in the size of the input formula and the number of edges in the input signal.
\end{claim}
\begin{proof}
	Let $\varepsilon$ be the maximum clock skew and $\delta$ the bounded variability constant.
	Let $n$ be the number of signals, $g$ be the number of edges of the signal, and $m$ be the size of the input formula.
		
	\begin{enumerate}
		\item 
		Enumerating the subformulas take $O(m)$ time and space.
		
		\item 
		Computing the canonical segmentation takes $O(gn)$ time and $O(g)$ space. % could do O(g logg) time too
		
		\item \ege{To be checked.}
		Computing the value expressions of signals takes \alert{$O(n g^3 3^g)$} time and \alert{$O(g^2 3^g)$} space.
		For each $j$ and $i$ computing the set $Y_{j,i}$ of uncertainty intervals takes $O(g)$ time and space.
		Computing each set $\operatorname{op}(v)$ takes constant time and space.
		Concatenating these sets takes $O(3^g)$ time and space, and it results in a set of value expressions of size $O(3^g)$ where each element is of length $O(g^2)$.
		Destuttering the final set takes $O(g^2 3^g)$ time and space.\\
		Note: the maximum number of uncertainty intervals that contain a canonical segment is $\frac{2\varepsilon}{\delta}$, which implies $|\gamma(I_j, i)| = O(4^{\frac{2\varepsilon}{\delta}})$ for all $j$ and $i$, and $|v| = O(\frac{2\varepsilon}{\delta})$ for all $v \in \gamma(I_j, i)$.
		
		\item \TODO:
		Computing the value expressions of subformulas takes $O(g m ...)$ time %O()
		and $O(...)$ space. %O()
		
		\item \TODO:
		Computing the output takes $O()$ time %O(g * numElts + g * maxLen)?
		and $O(...)$ space. %O(g * numElts * maxLen)? 
	\end{enumerate}
	
%	max num of uncertainty intervals that contain a canonical segment = $\frac{2\varepsilon}{\delta}$
%	
%	$|\gamma(I_j, i)| = O(4^{\frac{2\varepsilon}{\delta}})$
%	
%	$v \in \gamma(I_j, i) \implies |v| = O(\frac{2\varepsilon}{\delta})$
	
%	Computing the initial value expressions w.r.t. local uncertainty intervals takes $O(n g)$ time and space.
%	
%	Computing the initial value expressions $\gamma$ w.r.t. $G_S$ takes $O(n g \alert{2^{\frac{2\varepsilon}{\delta}}})$ time and space. \ege{To be checked.}
%	%for each signal : for each segment : for each element in the segment's current value expression (for general signals this is exponential in the number of uncertainty intervals containing the segment (which is at most 2epsilon/delta?), linear for alternating bits) : extend the current element in at most 3 ways (length of the value expression is 2, for infix we get 3 elements) =  O(n g 2^... 3)
%	
%	Bound on the lengths of boolean value expressions of atomic propositions: the same as that of general value expressions -- linear in the number of uncertainty intervals it is contained in, i.e., \alert{$O(\frac{2\varepsilon}{\delta})$}.
%	
%	Bound on the lengths of boolean value expressions of other subformulas: previous expression times depth of formula (due to blow up with $\land$) except for timed operators, then it is plus $O(\frac{b-a}{\delta})$.
%	
%	Bound on the size of boolean value expression sets of atomic propositions: linear in the number of uncertainty intervals it is contained in (instead of exponential as in the general case), i.e., \alert{$O(\frac{2\varepsilon}{\delta})$}, because here the value expressions are alternating bit sequences.
%	
%	Bound on the size of boolean value expression sets of other subformulas: 
\end{proof}