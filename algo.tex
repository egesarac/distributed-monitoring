\section{The Offline Algorithm}

Let $\varphi$ be an STL formula and $(S,{\hb})$ be a distributed signal of $n$ signals $x_1, \ldots x_n$ over the temporal domain $[0,d)$.

\begin{enumerate}
	\item 
	Enumerate the subformulas of $\varphi$ such that each formula has an enumeration number smaller than the numbers of all its subformulas.
	Let $\varphi_1, \ldots, \varphi_m$ be such an enumeration.
	Note that $\varphi_1 = \varphi$.
	
	\item
	Compute the canonical segmentation $G_{S} = \{ [t_1, t_2), \ldots, [t_k, t_{k+1}) \}$ of $(S, {\hb})$.
	
	\item
	Compute the value expressions of signals with respect to the canonical segmentation, i.e., for each $1 \leq j \leq k$ and $1 \leq i \leq n$, compute $\gamma([t_j, t_{j+1}), i)$.
	
	\item
	Compute the value expressions of subformulas with respect to the canonical segmentation, i.e., for each $1 \leq j \leq k$ and $1 \leq \ell \leq m$, compute $[S, t_j \models \varphi_\ell]$.
	
	\item
	Output the set $\mathsf{Out}(\varphi, S, {\hb}, \varepsilon, \delta) = \destutter([S,t_1 \models \varphi_1] \cdot \ldots \cdot [S,t_k \models \varphi_1])$ of value expressions. 
\end{enumerate}

\begin{theorem}
%	The algorithm is correct.
	

	For all STL formulas $\varphi$ and distributed signals $(S, {\hb})$ of bounded clock skew $\varepsilon$ and bounded variability $\delta$, we have $\mathsf{Goal}(\varphi, S, {\hb}, \varepsilon, \delta) \subseteq \mathsf{Out}(\varphi, S, {\hb}, \varepsilon, \delta)$.
\end{theorem}
\begin{proof}
	Let $\varphi$ be an STL formula, $(S, {\hb})$ be a distributed signal of bounded clock skew $\varepsilon$ and bounded variability $\delta$ with $S = (x_1, \ldots, x_n)$.
	
	Recall that each element of $\mathsf{Goal}(\varphi, S, {\hb}, \varepsilon, \delta)$ is a boolean value expression for the satisfaction signal of $\varphi$ for some trace of $(S,{\hb})$.
	Let $v \in \mathsf{Goal}(\varphi, S, {\hb}, \varepsilon, \delta)$ be arbitrary.
	To conclude that $v \in \mathsf{Out}(\varphi, S, {\hb}, \varepsilon, \delta)$, we argue that the semantics of our logic explores all admissible interleavings (subject to ${\hb}$, $\varepsilon$, and $\delta$) of the edges of $S$.
	
	First, notice that the canonical segmentation $G_S = \{ I_1, \ldots, I_k \}$ and the function $\gamma$ that maps each segment in $G_S$ and signal in $S$ to a value expression capture all potential behaviors of the individual signals in $S$.
	In particular, we claim that for each trace $w = (x_1', \ldots, x_n') \in \tr(S, {\hb}, \delta)$ and for each $i$, we have $\mu(x_i') \in \destutter(\gamma(I_1, i) \ldots \gamma(I_k, i))$.
	This holds because (i) each signal $x_i'$ has the same edges as $x_i$ modulo the time stamps, and (ii) for each $x_i$ and each uncertainty interval $J$ of $x_i$, the value expression of $J$ can be reconstructed by some choice of elements from the sequence of $\gamma$ values corresponding to the value expressions of the canonical segments that are contained in $J$.
	
	Now, we argue that the semantics of our logic preserve this property.
	One key observation is that the asynchronous product of two sets of value expressions captures all the admissible interleavings of the edges involved in each potential behavior encoded as a value expression these sets.
	Then, the cases of atomic propositions, negation, and conjunction are trivial.
	For untimed until, the definitions of the bitwise-temporal operations coincide with their discrete counterparts, therefore they are correct for each interleaving.
	For timed eventually and always, in addition to the arguments above, observe that the function $\mathsf{profilesBetween}$ enumerates all possible relations between the operator's interval $I$ and the segments we compute the satisfaction over.
	
%	For atomic propositions and conjunction, simply observe that the asynchronous product operation captures all the admissible interleavings of the edges involved.
%	For untimed until, observe that we consider the asynchronous product and apply bitwise-until operations whose definitions coincide with the usual recursive definitions.
%	For timed eventually and timed always, observe that the profiles of intervals capture all potential overlappings of the operation's interval and the canonical segments, to which we apply the corresponding bitwise operations which coincide with the usual definitions.
%	The case of negation is trivial.
\end{proof}

\begin{claim}
	\TODO: The algorithm runs in ... %time and space in the size of the input formula and the number of edges in the input signal.
\end{claim}
\begin{proof}
	Let $(S, {\hb})$ be a distributed signal of $n$ signals of length $d$ and a total of $g$ edges.
	Let $\varphi$ be an STL formula of size $m$.
	Let $\varepsilon$ be the maximum clock skew and $\delta$ the bounded variability constant.
	Note that $g \leq n  \lceil \frac{d + \varepsilon}{\delta} \rceil$.
		
	\begin{enumerate}
		\item 
%		Enumerating the subformulas take $O(m)$ time and space.
		Enumerating the subformulas take time and space linear in $m$.
		
		\item 
%		Computing the canonical segmentation takes $O(gn)$ time and $O(g)$ space: % could do O(g logg) time too
%		Knowing the edges of the signals, computing their uncertainty intervals takes $O(g)$ time in total and results in $O(g)$ values to be stored.
%		Then, merging the lists of values obtained in the previous step can be done in $O(gn)$ time.

%		Computing the canonical segmentation takes $O(g \log g)$ time and $O(g)$ space.
		Computing the canonical segmentation takes loglinear time and linear space in $g$.
		Knowing the edges of the signals, computing their uncertainty intervals takes $O(1)$ time each.
		We store their end points in a list of size $O(g)$, which we sort in $O(g \log g)$ time.
		
		\item
%		Computing the value expressions of signals takes $2^{O(g + \log n)}$ time and space:
%		Computing all the sets $Y_{k,i}$ of uncertainty intervals takes $O(gn)$ time and $O(g)$ space (we can piggyback this to the previous step).
%		Computing each set $\operatorname{op}(v)$ takes $O(1)$ time and space, and since each $Y_{k,i}$ contains $O(g)$ elements, % this seems unlikely in practice
%		it takes $O(g^2)$ time and space in total.
%		Note that each $\operatorname{op}(v)$ contains at most 4 elements of length at most 2.
%		Computing each set $Z_{k,i}$ takes $2^{O(g)}$ time and space due to concatenation (destuttering can be done on-the-fly with $O(1)$ overhead for each element of $Z_{k,i}$).
%		Note that each set $\gamma(I_k, i)$ contains $2^{O(g)}$ elements of length $O(g)$.
%		Since there are $n$ signals and $O(g)$ segments, there are $O(gn)$ such sets, and thus this step takes $2^{O(g \log n)}$ time and space. 
		%%$O(g) \cdot 2^{O(g)} = 2^{O(g)}$ and $O(n) \cdot 2^{O(g)} = 2^{O(g + \log n)}$
			
		
%		Computing the value expressions of signals takes $2^{O(\frac{\varepsilon}{\delta} + \log g + \log n)}$ time and space.
		Computing the value expressions of signals takes time and space linear in $g$ and $n$ and exponential in $O(\frac{\varepsilon}{\delta})$.
		Note that each set $Y_{k,i}$ has at most $\frac{4 \varepsilon}{\delta}$ elements. % which is less than $n  \lceil \frac{d + \varepsilon}{\delta} \rceil$ assuming $n \geq 2$ and $d > \varepsilon$.
		Computing all the sets $Y_{k,i}$ of uncertainty intervals takes $O(g^2)$ time.
		Note that each set $Y_{k,i}$ has at most $\frac{4 \varepsilon}{\delta}$ elements, so the computation takes $O(\frac{g \varepsilon}{\delta})$ space.
		Computing each set $\operatorname{op}(v)$ takes $O(1)$ time and space, and since each $Y_{k,i}$ contains $O(\frac{\varepsilon}{\delta})$ elements, it takes $O(\frac{g \varepsilon}{\delta})$ time and space in total.
		Note that each $\operatorname{op}(v)$ contains at most 4 elements of length at most 2, and there are $O(\frac{\varepsilon}{\delta})$ many of them for each $k$ and $i$.
		Computing each set $Z_{k,i}$ takes $2^{O(\frac{\varepsilon}{\delta})}$ time and space due to concatenation (destuttering can be done on-the-fly with $O(1)$ overhead for each element of $\operatorname{op}(v)$). %$Z_{k,i}$
		Note that each set $\gamma(I_k, i)$ contains $2^{O(\frac{\varepsilon}{\delta})}$ elements of length $O(\frac{\varepsilon}{\delta})$.
		Since there are $n$ signals and $O(g)$ segments, there are $O(gn)$ such sets, and thus this step takes $2^{O(\frac{\varepsilon}{\delta} + \log g + \log n)}$ time and space. 
	
%		Computing the value expressions of signals takes $...$ time and space:
%		Computing all the sets $Y_{k,i}$ of uncertainty intervals takes $O(\frac{n^2 (d + \varepsilon)}{\delta})$ time and $O(\frac{n (d + \varepsilon)}{\delta})$ space (we can piggyback this to the previous step).
%		Computing each set $\operatorname{op}(v)$ takes $O(1)$ time and space, and since each $Y_{k,i}$ contains $O(g)$ elements, % this seems unlikely in practice
%		it takes $O(g^2)$ time and space in total.
%		Note that each $\operatorname{op}(v)$ contains at most 4 elements of length at most 2.
%		Computing each set $Z_{k,i}$ takes $2^{O(g)}$ time and space due to concatenation (destuttering can be done on-the-fly with $O(1)$ overhead for each element of $Z_{k,i}$).
%		Note that each set $\gamma(I_k, i)$ contains $2^{O(g)}$ elements of length $O(g)$.
%		Since there are $n$ signals and $O(g)$ segments, there are $O(gn)$ such sets, and thus this step takes $2^{O(g \log n)}$ time and space.
		
%		Computing the value expressions of signals takes \alert{$O(n g^3 3^g)$} time and \alert{$O(g^2 3^g)$} space.
%		For each $j$ and $i$ computing the set $Y_{j,i}$ of uncertainty intervals takes $O(g)$ time and space.
%		Computing each set $\operatorname{op}(v)$ takes constant time and space.
%		Concatenating these sets takes $O(3^g)$ time and space, and it results in a set of value expressions of size $O(3^g)$ where each element is of length $O(g^2)$.
%		Destuttering the final set takes $O(g^2 3^g)$ time and space.\\
%		Note: the maximum number of uncertainty intervals that contain a canonical segment is $\frac{2\varepsilon}{\delta}$, which implies $|\gamma(I_j, i)| = O(4^{\frac{2\varepsilon}{\delta}})$ for all $j$ and $i$, and $|v| = O(\frac{2\varepsilon}{\delta})$ for all $v \in \gamma(I_j, i)$.
		
		
		\item \TODO:
		Computing the value expressions of subformulas takes $O(...)$ time %O(g m ...)
		and $O(...)$ space. %O()
		
		
		For an atomic proposition $p$, we take as input the $\gamma$ sets of each segment and the signals involved in $p$.
		Recall from the previous step that each $\gamma$ contains $2^{O(\frac{\varepsilon}{\delta})}$ elements of length $O(\frac{\varepsilon}{\delta})$.
		Then, the set of value expressions of an atomic proposition, which is a function of $n$ signals, contains elements of length $O(\frac{n\varepsilon}{\delta})$ due to the asynchronous product, and $O(\frac{n\varepsilon}{\delta})$ elements due to destuttering.
		Note that the asynchronous product of $n$ signals contains $2^{O(\frac{n\varepsilon}{\delta})}$ elements of length $O(\frac{n\varepsilon}{\delta})$, therefore the computation takes $2^{O(\frac{n\varepsilon}{\delta})}$ time and space. %% check
		
		We investigate the remaining operations where the input is set(s) of boolean value expressions of $\alpha$ elements of length at most $\beta$.
%		Note that the asynchronous product of two such sets results in expressions of length $O(\beta)$, which corresponds to a number of elements of $O(\beta)$ due to destuttering.
		
		For negation, it is easy to see that the computation takes $O(\alpha \beta)$ time and space since we iterate over each bit and negate.
		The resulting set of value expressions has $\alpha$ elements of length at most $\beta$.
		%TODO: there is a more efficient way using the 2d-box representation. note this somewhere and check all the statements.
		
%		For conjunction, for each pair in the asynchronous product we compute the corresponding set of value expressions in time 
%		This part of the computation takes $O(\alpha^2 \)$ time and $O(\beta)$ space. 
%		Due to the asynchronous product, 
		
		
		
		
		
		
		
		
		\item \TODO:
		Computing the output takes $O(...)$ time %O(g * numElts + g * maxLen)?
		and $O(...)$ space. %O(g * numElts * maxLen)? 
	\end{enumerate}
	
%	max num of uncertainty intervals that contain a canonical segment = $\frac{2\varepsilon}{\delta}$
%	
%	$|\gamma(I_j, i)| = O(4^{\frac{2\varepsilon}{\delta}})$
%	
%	$v \in \gamma(I_j, i) \implies |v| = O(\frac{2\varepsilon}{\delta})$
	
%	Computing the initial value expressions w.r.t. local uncertainty intervals takes $O(n g)$ time and space.
%	
%	Computing the initial value expressions $\gamma$ w.r.t. $G_S$ takes $O(n g \alert{2^{\frac{2\varepsilon}{\delta}}})$ time and space. \ege{To be checked.}
%	%for each signal : for each segment : for each element in the segment's current value expression (for general signals this is exponential in the number of uncertainty intervals containing the segment (which is at most 2epsilon/delta?), linear for alternating bits) : extend the current element in at most 3 ways (length of the value expression is 2, for infix we get 3 elements) =  O(n g 2^... 3)
%	
%	Bound on the lengths of boolean value expressions of atomic propositions: the same as that of general value expressions -- linear in the number of uncertainty intervals it is contained in, i.e., \alert{$O(\frac{2\varepsilon}{\delta})$}.
%	
%	Bound on the lengths of boolean value expressions of other subformulas: previous expression times depth of formula (due to blow up with $\land$) except for timed operators, then it is plus $O(\frac{b-a}{\delta})$.
%	
%	Bound on the size of boolean value expression sets of atomic propositions: linear in the number of uncertainty intervals it is contained in (instead of exponential as in the general case), i.e., \alert{$O(\frac{2\varepsilon}{\delta})$}, because here the value expressions are alternating bit sequences.
%	
%	Bound on the size of boolean value expression sets of other subformulas: 
\end{proof}