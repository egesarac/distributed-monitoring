\section{The Offline Algorithm}

Let $\varphi$ be an STL formula and $(S,{\hb})$ be a distributed signal of $n$ signals $x_1, \ldots x_n$ over the temporal domain $[0,d)$.

\begin{enumerate}
	\item 
	Enumerate the subformulas of $\varphi$ such that each formula has an enumeration number smaller than the numbers of all its subformulas.
	Let $\varphi_1, \ldots, \varphi_m$ be such an enumeration.
	Note that $\varphi_1 = \varphi$.
	
	\item
	Let $G_{S} = \{ [t_1, t_2), \ldots, [t_k, t_{k+1}) \}$ be the canonical segmentation of $(S, {\hb})$.
	
	\item
	For $i = k .. 1$, for $j = m .. 1$, compute $[S, t_i \models \varphi_j]$.
	
	\item
	Output the set $\mathsf{Out}(\varphi, S, {\hb}, \varepsilon, \delta) = \destutter([S,t_1 \models \varphi_1] \cdot \ldots \cdot [S,t_k \models \varphi_1])$ of value expressions. %\TODO: match the semantics -- \alert{$\first([S',0 \models \varphi_1'])$}.
\end{enumerate}

\begin{claim}
	The algorithm is correct.
	For all STL formulas $\varphi$, and distributed signals $(S, {\hb})$ of bounded clock skew $\varepsilon$ and bounded variability $\delta$, we have $\mathsf{Goal}(\varphi, S, {\hb}, \varepsilon, \delta) \subseteq \mathsf{Out}(\varphi, S, {\hb}, \varepsilon, \delta)$.
\end{claim}
\begin{proof}
	\TODO: easy to see by the semantics of value expressions
\end{proof}

\begin{claim}
	\TODO: The algorithm runs in ... %time and space in the size of the input formula and the number of edges in the input signal.
\end{claim}
\begin{proof}
	\TODO %first bound the length of value expressions using bounded variability
	
	Let $(S {\hb})$ be a distributed signal of $n$ signals.
	Let $\varepsilon$ be the maximum clock skew and $\delta$ the bounded variability constant.
	Let $g$ be the number of edges of the signal.
	
	Computing the canonical segmentation $G_S$ takes $O(g (n + \log g))$ time and $O(g)$ space.
	
	Computing the initial value expressions w.r.t. local uncertainty intervals takes $O(n g)$ time and space.
	
	Computing the initial value expressions $\gamma$ w.r.t. $G_S$ takes $O(n g \alert{2^{\frac{2\varepsilon}{\delta}}})$ time and space. \ege{To be checked.}
	%for each signal : for each segment : for each element in the segment's current value expression (for general signals this is exponential in the number of uncertainty intervals containing the segment (which is at most 2epsilon/delta?), linear for alternating bits) : extend the current element in at most 3 ways (length of the value expression is 2, for infix we get 3 elements) =  O(n g 2^... 3)
	
	Bound on the lengths of boolean value expressions of atomic propositions: the same as that of general value expressions -- linear in the number of uncertainty intervals it is contained in, i.e., \alert{$O(\frac{2\varepsilon}{\delta})$}.
	
	Bound on the lengths of boolean value expressions of other subformulas: previous expression times depth of formula (due to blow up with $\land$) except for timed operators, then it is plus $O(\frac{b-a}{\delta})$.
	
	Bound on the size of boolean value expression sets of atomic propositions: linear in the number of uncertainty intervals it is contained in (instead of exponential as in the general case), i.e., \alert{$O(\frac{2\varepsilon}{\delta})$}, because here the value expressions are alternating bit sequences.
	
	Bound on the size of boolean value expression sets of other subformulas: 
	
	
\end{proof}