\section{The Offline Algorithm}

Let $\varphi$ be an STL formula and $(S,{\hb})$ be a distributed signal of $n$ signals $x_1, \ldots x_n$ over the temporal domain $[0,d)$.

\begin{enumerate}
	\item 
	Enumerate the subformulas of $\varphi$ such that each formula has an enumeration number smaller than the numbers of all its subformulas.
	Let $\varphi_1, \ldots, \varphi_m$ be such an enumeration.
	Note that $\varphi_1 = \varphi$.
	
	\item
	Compute the canonical segmentation $G_{S} = \{ [t_1, t_2), \ldots, [t_k, t_{k+1}) \}$ of $(S, {\hb})$.
	
	\item
	Compute the value expressions of signals with respect to the canonical segmentation, i.e., for each $1 \leq j \leq k$ and $1 \leq i \leq n$, compute $\gamma([t_j, t_{j+1}), i)$.
	
	\item
	Compute the value expressions of subformulas with respect to the canonical segmentation, i.e., for each $1 \leq j \leq k$ and $1 \leq \ell \leq m$, compute $[S, t_j \models \varphi_\ell]$.
	
	\item
	Output the set $\mathsf{Out}(\varphi, S, {\hb}, \varepsilon, \delta) = \destutter([S,t_1 \models \varphi_1] \cdot \ldots \cdot [S,t_k \models \varphi_1])$ of value expressions. 
\end{enumerate}

\begin{theorem}
%	The algorithm is correct.
	

	For all STL formulas $\varphi$ and distributed signals $(S, {\hb})$ of bounded clock skew $\varepsilon$ and bounded variability $\delta$, we have $\mathsf{Goal}(\varphi, S, {\hb}, \varepsilon, \delta) \subseteq \mathsf{Out}(\varphi, S, {\hb}, \varepsilon, \delta)$.
\end{theorem}
\begin{proof}
	Let $\varphi$ be an STL formula, $(S, {\hb})$ be a distributed signal of bounded clock skew $\varepsilon$ and bounded variability $\delta$ with $S = (x_1, \ldots, x_n)$.
	
	Recall that each element of $\mathsf{Goal}(\varphi, S, {\hb}, \varepsilon, \delta)$ is a boolean value expression for the satisfaction signal of $\varphi$ for some trace of $(S,{\hb})$.
	Let $v \in \mathsf{Goal}(\varphi, S, {\hb}, \varepsilon, \delta)$ be arbitrary.
	To conclude that $v \in \mathsf{Out}(\varphi, S, {\hb}, \varepsilon, \delta)$, we argue that the semantics of our logic explores all admissible interleavings (subject to ${\hb}$, $\varepsilon$, and $\delta$) of the edges of $S$.
	
	First, notice that the canonical segmentation $G_S = \{ I_1, \ldots, I_k \}$ and the function $\gamma$ that maps each segment in $G_S$ and signal in $S$ to a value expression capture all potential behaviors of the individual signals in $S$.
	In particular, we claim that for each trace $w = (x_1', \ldots, x_n') \in \tr(S, {\hb}, \delta)$ and for each $i$, we have $\mu(x_i') \in \destutter(\gamma(I_1, i) \ldots \gamma(I_k, i))$.
	This holds because (i) each signal $x_i'$ has the same edges as $x_i$ modulo the time stamps, and (ii) for each $x_i$ and each uncertainty interval $J$ of $x_i$, the value expression of $J$ can be reconstructed by some choice of elements from the sequence of $\gamma$ values corresponding to the value expressions of the canonical segments that are contained in $J$.
	
	Now, we argue that the semantics of our logic preserve this property.
	One key observation is that the asynchronous product of two sets of value expressions captures all the admissible interleavings of the edges involved in each potential behavior encoded as a value expression these sets.
	Then, the cases of atomic propositions, negation, and conjunction are trivial.
	For untimed until, the definitions of the bitwise-temporal operations coincide with their discrete counterparts, therefore they are correct for each interleaving.
	For timed eventually and always, in addition to the arguments above, observe that the function $\mathsf{profilesBetween}$ enumerates all possible relations between the operator's interval $I$ and the segments we compute the satisfaction over.
	
%	For atomic propositions and conjunction, simply observe that the asynchronous product operation captures all the admissible interleavings of the edges involved.
%	For untimed until, observe that we consider the asynchronous product and apply bitwise-until operations whose definitions coincide with the usual recursive definitions.
%	For timed eventually and timed always, observe that the profiles of intervals capture all potential overlappings of the operation's interval and the canonical segments, to which we apply the corresponding bitwise operations which coincide with the usual definitions.
%	The case of negation is trivial.
\end{proof}

\begin{claim}
	\TODO: The algorithm runs in ... %time and space in the size of the input formula and the number of edges in the input signal.
\end{claim}
\begin{proof}
	Let $(S, {\hb})$ be a distributed signal of $n$ signals of length $d$ and a total of $g$ edges.
	Let $\varphi$ be an STL formula of size $m$.
	Let $\varepsilon$ be the maximum clock skew and $\delta$ the bounded variability constant.
	Note that $g \leq n  \lceil \frac{d + \varepsilon}{\delta} \rceil$.
		
	\begin{enumerate}
		\item 
		\textbf{Enumerating the subformulas take $O(m)$ time and space.}
		
		\item 
		\textbf{Computing the canonical segmentation takes $O(g \log g)$ time and $O(g)$ space.}
		\begin{itemize}
			\item 
			Given the signals' edges, computing their uncertainty intervals takes $O(1)$ time for each edge.
			We store their end points in a list of size $O(g)$, which we sort in $O(g \log g)$ time.
		\end{itemize}
		
		
		
		\item	
		\textbf{Computing the value expressions of signals takes $2^{O(\frac{\varepsilon}{\delta} + \log g + \log n)}$ time and space.}
		\begin{itemize}
			\item
			For each segment $I_k$, we compute the sets $Y_{k,1}, \ldots Y_{k,n}$ by searching the end points of $I_k$ in the uncertainty intervals of $x_1, \ldots, x_n$ one by one, which takes $O(g)$ time.
			Since there are $O(g)$ segments, computing all the sets $Y_{k,i}$ takes $O(g^2)$ time.
			Note that each set $Y_{k,i}$ has at most $\frac{4 \varepsilon}{\delta}$ elements, so the computation takes $O(\frac{g n \varepsilon}{\delta})$ space.
			
			\item
			Computing each set $\operatorname{op}(v)$ takes $O(1)$ time and space.
			Note that each $\operatorname{op}(v)$ contains at most 4 elements of length at most 2.
			Since each $Y_{k,i}$ contains $O(\frac{\varepsilon}{\delta})$ elements, there are $O(\frac{\varepsilon}{\delta})$ many $\operatorname{op}(v)$ sets for each $k$ and $i$.
			Therefore, it takes $O(\frac{g n \varepsilon}{\delta})$ time and space to compute all the sets $\operatorname{op}(v)$.
			
			\item
			To compute each set $Z_{k,i}$, we take concatenate and destutter the $\operatorname{op}(v)$ sets.
			The concatenation takes $2^{O(\frac{\varepsilon}{\delta})}$ time and space because each $\operatorname{op}(v)$ set has $O(1)$ elements of $O(1)$ length, and there are $O(\frac{\varepsilon}{\delta})$ such sets.
			Note that each set $Z_{k,i}$ has $2^{O(\frac{\varepsilon}{\delta})}$ elements of length $O(\frac{\varepsilon}{\delta})$.
			
			\item
			For computing each set $\gamma(I_k, i)$ from $Z_{k,i}$, the desuttering can be done on-the-fly in the previous step by comparing the first and last elements of the concatenated expressions, which takes $O(1)$ time and space for each element of each $\operatorname{op}(v)$ set.
			Finally, since there are $n$ signals and $O(g)$ segments, computing all the sets $\gamma(I_k, i)$ takes $2^{O(\frac{\varepsilon}{\delta} + \log g + \log n)}$ time and space.
		\end{itemize}

		\item
		\textbf{Computing the value expressions of subformulas takes $O(...)$ time and $O(...)$ space.}
		\begin{itemize}
			\item
			For an atomic proposition $p$, we take as input the $\gamma$ sets of each segment and the signals involved in $p$.
			Recall from the previous step that each $\gamma$ contains $2^{O(\frac{\varepsilon}{\delta})}$ elements of length $O(\frac{\varepsilon}{\delta})$.
			To compute the corresponding $n$-ary function $f_p$ in a given segment $I_k$, we need to compute the asynchronous product $\gamma(I_k, 1) \otimes \ldots \otimes \gamma(I_k, n)$.
			This computation takes $2^{O(\frac{n\varepsilon}{\delta})}$ time and space, and the product contains $2^{O(\frac{n\varepsilon}{\delta})}$ elements of length $O(\frac{n\varepsilon}{\delta})$.
			We assume that, for a single-letter alignment of the tuples in the product, applying $f_p$ and comparing the output with zero take $O(1)$ time, and thus $2^{O(\frac{n\varepsilon}{\delta})}$ time for a single segment.
			During these computations, we compute the boolean value expressions by destuttering on the fly.
			Repeating the steps for each segment takes $2^{O(\frac{n\varepsilon}{\delta} + \log g)}$ time, resulting in $O(g)$ sets of boolean value expressions, each containing $O(\frac{n\varepsilon}{\delta})$ elements of length $O(\frac{n\varepsilon}{\delta})$.
			
			\item 
			We investigate the remaining operations where we take the inputs as sets of boolean value expressions of $O(\alpha)$ elements of length $O(\beta)$.
			We remark that a set of boolean value expressions can be represented as two bit vectors marking the positions corresponding to elements in the set.
			For example, the set $U = \{0, 01, 10, 101\}$ can be represented as $u_0 = 011$ and $u_1 = 110$ where $u_\sigma$ encodes the elements that start with letter $\sigma$ and a position $i$ is marked with 1 iff $U$ contains an element of length $i$ that starts with $\sigma$.
			
			\item 
			To compute the negation of a set of value expressions, we simply store a boolean flag that effectively swaps the names of the corresponding bit vectors.
			Since this flag applies to all segments at once, it takes $O(1)$ time and space in total.
						
			\item
			For the conjunction of two sets of value expressions, first observe the following:
			(i) for all $u,v \in \destutter(\Sigma^*)$, we can compute $u \land v$ in $O(|u|+|v|)$ time and space, and
			(ii) for all $u, u', v, v' \in \destutter(\Sigma^*)$ such that $u$ and $u'$ start and end with the same letter, $v$ and $v'$ start and end with the same letter, $|u'| \leq |u|$ and $|v'| \leq |v|$, we have $u' \land v' \subseteq u \land v$.
			These statements hold because the conjunction of two boolean value expressions cannot have more edges can the sum of the number of edges of the inputs, and they can be aligned to obtain an output with any smaller number of edges. \ege{There are some corner cases to consider, but this is almost true.}
			Therefore, we only need to compute the conjunctions of the longest elements of each type in each set (if they exist), which takes $O(\beta_1 + \beta_2)$ time and space for each segment, and thus $O(g (\beta_1 + \beta_2))$ time and space in total.
			The resulting sets each contain $O(\beta_1 + \beta_2)$ elements of length $O(\beta_1 + \beta_2)$.
			Note that each conjunction potentially doubles the set size.
			
			\item
			For the until operator, observe that 
			(i) for all $u,v \in \destutter(\Sigma^*)$, we can compute $u \land v$ in $O(|v|)$ time and space, and
			(ii) for all $u, u', v, v' \in \destutter(\Sigma^*)$ such that $u$ and $u'$ start and end with the same letter, $v$ and $v'$ start and end with the same letter, $|u'| \leq |u|$ and $|v'| \leq |v|$, we have $u' \until v' \subseteq u \until v$.
			These hold for similar reasons as for conjunction, but now the number of edges in the output depends on the number of edges of $v$ and not of $u$. \ege{To be checked.}
			Therefore, the computation takes $O(\beta_2)$ time and space for each segment, and thus $O(g \beta_2)$ time and space in total.
			The resulting sets each contain $O(\beta_2)$ elements of length $O(\beta_2)$.
			
			\item
			For the bounded temporal operators, first note that there are $O(g)$ profiles to be considered.
			Computing a profile may include a concatenation of $O(g)$ sets of value expressions (due to $\kappa$ when $I$ is large), thus it takes $2^{O(g \log \alpha)}$ time and space, resulting in a set of $O(\alpha (g + \beta))$ elements of length $O(\alpha (g + \beta))$.
			Computing the bitwise temporal operators on such a set takes $O(\alpha (g + \beta))$ time, resulting in a set of $O(1)$ elements of length $O(1)$.
			Concatenating the resulting sets of $O(g)$ profiles thus takes $O(g)$ time and space, resulting in a set of $O(g)$ elements of length $O(g)$.			
			
			\item
			\TODO: explain the worst case and update the statement.
			
			
%			The worst case occurs when the input formula consists of conjunctions only. \ege{To be checked.}
%			This leads to $O(\frac{m n \epsilon}{\delta})$ work at each level of the syntax tree of the formula.
%			Therefore, in total for $O(\log m)$ levels and $O(g)$ segments, it takes 
			
%			This leads to doubling the set size and the expression length at each level, resulting at the root in an expression of size 
		\end{itemize}

		\item
		\textbf{Computing the output takes $2^{O(g \log X)}$ time and $O(g Y)$ space.}
		We concatenate $O(g)$ sets of value expressions for the root formula, which contains in the worst case $O(X)$ elements of length $O(Y)$.
		This can be done together with on-the-fly destuttering in $2^{O(g \log X)}$ time and $O(g Y)$ space.
		\TODO: replace $X$ and $Y$ based on the worst case above.

	\end{enumerate}
	
%	max num of uncertainty intervals that contain a canonical segment = $\frac{2\varepsilon}{\delta}$
%	
%	$|\gamma(I_j, i)| = O(4^{\frac{2\varepsilon}{\delta}})$
%	
%	$v \in \gamma(I_j, i) \implies |v| = O(\frac{2\varepsilon}{\delta})$
	
%	Computing the initial value expressions w.r.t. local uncertainty intervals takes $O(n g)$ time and space.
%	
%	Computing the initial value expressions $\gamma$ w.r.t. $G_S$ takes $O(n g \alert{2^{\frac{2\varepsilon}{\delta}}})$ time and space. \ege{To be checked.}
%	%for each signal : for each segment : for each element in the segment's current value expression (for general signals this is exponential in the number of uncertainty intervals containing the segment (which is at most 2epsilon/delta?), linear for alternating bits) : extend the current element in at most 3 ways (length of the value expression is 2, for infix we get 3 elements) =  O(n g 2^... 3)
%	
%	Bound on the lengths of boolean value expressions of atomic propositions: the same as that of general value expressions -- linear in the number of uncertainty intervals it is contained in, i.e., \alert{$O(\frac{2\varepsilon}{\delta})$}.
%	
%	Bound on the lengths of boolean value expressions of other subformulas: previous expression times depth of formula (due to blow up with $\land$) except for timed operators, then it is plus $O(\frac{b-a}{\delta})$.
%	
%	Bound on the size of boolean value expression sets of atomic propositions: linear in the number of uncertainty intervals it is contained in (instead of exponential as in the general case), i.e., \alert{$O(\frac{2\varepsilon}{\delta})$}, because here the value expressions are alternating bit sequences.
%	
%	Bound on the size of boolean value expression sets of other subformulas: 
\end{proof}