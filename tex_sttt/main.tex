%Version 3.1 December 2024
% See section 11 of the User Manual for version history
%
%%%%%%%%%%%%%%%%%%%%%%%%%%%%%%%%%%%%%%%%%%%%%%%%%%%%%%%%%%%%%%%%%%%%%%
%%                                                                 %%
%% Please do not use \input{...} to include other tex files.       %%
%% Submit your LaTeX manuscript as one .tex document.              %%
%%                                                                 %%
%% All additional figures and files should be attached             %%
%% separately and not embedded in the \TeX\ document itself.       %%
%%                                                                 %%
%%%%%%%%%%%%%%%%%%%%%%%%%%%%%%%%%%%%%%%%%%%%%%%%%%%%%%%%%%%%%%%%%%%%%

%%\documentclass[referee,sn-basic]{sn-jnl}% referee option is meant for double line spacing

%%=======================================================%%
%% to print line numbers in the margin use lineno option %%
%%=======================================================%%

%%\documentclass[lineno,pdflatex,sn-basic]{sn-jnl}% Basic Springer Nature Reference Style/Chemistry Reference Style

%%=========================================================================================%%
%% the documentclass is set to pdflatex as default. You can delete it if not appropriate.  %%
%%=========================================================================================%%

\documentclass[sn-nature,iicol,lineno]{sn-jnl}% Basic Springer Nature Reference Style/Chemistry Reference Style

%%Note: the following reference styles support Namedate and Numbered referencing. By default the style follows the most common style. To switch between the options you can add or remove “Numbered” in the optional parenthesis. 
%%The option is available for: sn-basic.bst, sn-chicago.bst%  

%\documentclass[pdflatex,sn-nature]{sn-jnl}% Style for submissions to Nature Portfolio journals
%\documentclass[pdflatex,sn-basic]{sn-jnl}% Basic Springer Nature Reference Style/Chemistry Reference Style
%\documentclass[pdflatex,sn-mathphys-num]{sn-jnl}% Math and Physical Sciences Numbered Reference Style
%%\documentclass[pdflatex,sn-mathphys-ay]{sn-jnl}% Math and Physical Sciences Author Year Reference Style
%%\documentclass[pdflatex,sn-aps]{sn-jnl}% American Physical Society (APS) Reference Style
%%\documentclass[pdflatex,sn-vancouver-num]{sn-jnl}% Vancouver Numbered Reference Style
%%\documentclass[pdflatex,sn-vancouver-ay]{sn-jnl}% Vancouver Author Year Reference Style
%%\documentclass[pdflatex,sn-apa]{sn-jnl}% APA Reference Style
%%\documentclass[pdflatex,sn-chicago]{sn-jnl}% Chicago-based Humanities Reference Style

\usepackage[utf8]{inputenc}
\usepackage[T1]{fontenc}
\usepackage{lmodern}
\usepackage[hidelinks]{hyperref}
\usepackage{diagbox}
\usepackage{todonotes}
\usepackage{ltl}
\usepackage{amssymb}
\usepackage{amsmath}
\usepackage{stmaryrd}
\usepackage{tikz}
\usetikzlibrary{backgrounds, positioning, arrows, calc}
\usepackage{mathtools}
\usepackage[linesnumbered,vlined,noend, ruled]{algorithm2e}
\usepackage[inline]{enumitem}  
\usepackage{cleveref}
\usepackage{lineno}
\linenumbers
\input{macros}

%% as per the requirement new theorem styles can be included as shown below
%\theoremstyle{thmstyleone}%
\newtheorem{theorem}{Theorem}%  meant for continuous numbers
%%\newtheorem{theorem}{Theorem}[section]% meant for sectionwise numbers
%% optional argument [theorem] produces theorem numbering sequence instead of independent numbers for Proposition
\newtheorem{lemma}{Lemma}% 
%%\newtheorem{proposition}{Proposition}% to get separate numbers for theorem and proposition etc.

%\theoremstyle{thmstyletwo}%
\newtheorem{example}{Example}%
\newtheorem{remark}{Remark}%

%\theoremstyle{thmstylethree}%
\newtheorem{definition}{Definition}%


\raggedbottom
%%\unnumbered% uncomment this for unnumbered level heads

\begin{document}

	\title[Approximate Distributed Monitoring under Partial Synchrony]{Approximate Distributed Monitoring under Partial Synchrony: Balancing Speed \& Accuracy}
	
	\author[1]{\fnm{Borzoo} \sur{Bonakdarpour}}\email{borzoo@msu.edu}
	\author[1]{\fnm{Anik} \sur{Momtaz}}\email{momtazan@msu.edu}
	\author[2]{\fnm{Dejan} \sur{Ni\v{c}kovi\'{c}}}\email{dejan.nickovic@ait.ac.at}
	\author*[3]{\fnm{N. Ege} \sur{Sara\c{c}}}\email{esarac@ista.ac.at}
	\affil[1]{\orgname{Michigan State University}}
	\affil[2]{\orgname{AIT Austrian Institute of Technology}}
	\affil*[3]{\orgname{Institute of Science and Technology Austria (ISTA)}}
	
	%%\pacs[JEL Classification]{D8, H51}
	
	%%\pacs[MSC Classification]{35A01, 65L10, 65L12, 65L20, 65L70}
	
	\maketitle
		
\vspace{-9mm}

\begin{abstract}
	
	In distributed systems with processes that do not share a global clock, \emph{partial synchrony} is 
	achieved by clock synchronization 
	that guarantees bounded clock skew among all applications. Existing solutions for distributed 
	runtime verification under partial 
	synchrony against temporal logic specifications are exact but suffer from significant 
	computational overhead. In this paper, we propose an \emph{approximate} 
	distributed monitoring algorithm for Signal Temporal Logic (STL) that mitigates this issue by 
	abstracting away potential interleaving behaviors. 
	This conservative abstraction enables a significant speedup of the distributed monitors, albeit 
	with a tradeoff in accuracy. We address this 
	tradeoff with a methodology that combines our approximate monitor with its exact counterpart, 
	resulting in enhanced monitoring 
	efficiency without sacrificing precision. We evaluate our approach with multiple 
	experiments, 
	showcasing its effectiveness and efficacy in 
	both real-world applications and synthetic examples.
	
\end{abstract}

%\borzoo{Alternative title: Sacrificing a Little Accuracy Significantly Improves Performance of 
%Distributed Monitoring}
	\input introduction
	\input preliminaries
	\input approach
	\input algorithm
	\input experiments
	\input conclusion
	
%	\bibliographystyle{spbasic}
	\bibliography{main}
%	\input main.bbl
	
	\newpage
	\input appendix
	
\end{document}