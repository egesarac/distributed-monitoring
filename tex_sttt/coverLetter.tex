\documentclass[10pt]{letter}
\usepackage{graphicx}
\usepackage{enumitem}
\usepackage{color}

\oddsidemargin=0in %.2in
\evensidemargin=.2in
\textwidth=6.5in
\topmargin=-1in
\textheight=9in

%% \pagestyle{empty}
\setlength{\textheight}{10.0in} %{8.875in}
%% \setlength{\columnsep}{2.5pc}
%% \setlength{\textwidth}{6.875in}
%% \setlength{\footheight}{0.0in}
%% \setlength{\topmargin}{0.0in}
%% \setlength{\oddsidemargin}{-0.25in}


\name{\vspace*{-1.5cm}N. Ege  Sara\c{c} \\
	Corresponding Author  \\ 
	\vspace*{0.5cm}
	\hspace*{-0.3cm}
	\begin{tabular}{lcl}
		Address & : & Institute of Science and Technology Austria \\
		&& Am Campus 1, 3400\\
		&& Klosterneuburg, Austria\\
		e-mail  & : & ege.sarac@ista.ac.at\\
	\end{tabular}
}
\date{\today}

\begin{document}
	\begin{letter}
		{
			Prof. Abraham and Prof. Abbas\\
			Guest Editors,\\
			International Journal on Software Tools for Technology Transfer
		}
		
		\opening{Dear Prof. Abraham and Prof. Abbas,}
		
		I hereby submit the manuscript ``Approximate Distributed Monitoring under Partial Synchrony: Balancing Speed \& Accuracy'' co-authored by Borzoo Bonakdarpour, Anik Momtaz, Dejan Ni\v{c}kovi\'{c}, and N. Ege Sara\c{c} to be considered for publication in the STTT Special Issue for RV 2024.
		
		The contribution of the manuscript can be summarized as follows:
		\setlist{nolistsep}
		\begin{itemize}[noitemsep]
			\item We introduce a conservative abstraction for distributed runtime verification under partial synchrony that overapproximates interleavings, and thus enables a sound approximate semantics for STL.
			\item We develop monitoring algorithms that compute the overapproximate semantics: an offline algorithm for both future and past STL, and online algorithms (only past STL) that operate on the incrementally available portions of the trace.
			\item We present implementation techniques (bit-vector–based representation, asynchronous products, and profile computation for timed operators) and a method to combine approximate and exact SMT-based monitors to preserve precision while improving efficiency.
			\item We provide a prototype tool and an extensive empirical evaluation on synthetic traces and two real-world case studies (water distribution system and swarm of drones), showing speedups of several orders of magnitude over exact monitors and characterizing when accuracy remains high.
		\end{itemize}
		
		The manuscript is original, has not been simultaneously submitted to a journal, and it extends the following conference paper:
		\begin{itemize}
			\item Borzoo Bonakdarpour, Anik Momtaz, Dejan Ni\v{c}kovi\'{c}, and N. Ege Sara\c{c}. Approximate Distributed Monitoring under Partial Synchrony: Balancing Speed \& Accuracy. In Erika Ábrahám and Houssam Abbas, editors, \emph{Runtime Verification - 24th International Conference, RV 2024, Istanbul, Turkey, October 15-17, 2024, Proceedings, volume 15191 of Lecture Notes in Computer Science}, pages 282–301. Springer, 2024.
		\end{itemize}
				
		
		Compared to the conference version, this manuscript includes the following new material:
		\begin{itemize}
			\item Two approximate algorithms for online monitoring of past-time STL: a naive baseline and an incremental algorithm with constant time per portion (for fixed portion size).
			\item Formal correctness proofs for both online algorithms.
			\item Experimental evaluation of the two online algorithms.
			\item Extension of the offline approximate monitor to past-time STL.
			\item Formal correctness proof for the offline algorithm (moved from the appendix).
			\item Additional experimental evaluation for the offline algorithm (moved from the appendix).
		\end{itemize}
		Thank you for your consideration. 
		
		\closing{Best regards,} %\\ \includegraphics[width=0.05\textwidth]{sign.png}
	\end{letter}
\end{document}