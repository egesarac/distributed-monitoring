\section{Preliminaries}

We define $\B = \{ \bot, \top \}$ as the set of boolean truth values where $\bot < \top$ and they complement each other.
We denote by $\R$ the set of reals, $\R_{\geq 0}$ the set of nonnegative reals, and $\R_{> 0}$ the set of positive reals.
A left-closed right-open interval $[a,b) \subseteq \R$ of reals has length $|b-a|$.
%, $\N$ the set of natural numbers (including 0).

Let $\Sigma$ be a finite alphabet.
We denote by $\Sigma^*$ the set of finite words over $\Sigma$ and by $\epsilon$ the empty word.
For $u \in \Sigma^*$, we respectively write $\pfx(u)$ and $\sfx(u)$ for the sets of prefixes and suffixes of $u$.
Moreover, we let $\infx(u) = \{v \in \Sigma^* \st \exists x,y \in \Sigma^* : u = xvy\}$.
For a nonempty word $u \in \Sigma^*$ and $1 \leq i \leq |u|$ we denote by $u[i]$ the $i$th letter of $u$, by $u[..i]$ the prefix of $u$ of length $i$, and by $u[i..]$ the suffix of $u$ of length $|u| - i + 1$. 

We define the function $\destutter : \Sigma^* \to \Sigma^*$ inductively as follows.
For all $\sigma \in \Sigma \cup \{\epsilon\}$, let $\destutter(\sigma) = \sigma$.
For all $u \in \Sigma^*$ such that $u = \sigma_1 \sigma_2 v$ for some $\sigma_1,\sigma_2 \in \Sigma$ and $v \in \Sigma^*$, let (i) $\destutter(u) = \destutter(\sigma_2 v)$ if $\sigma_1 = \sigma_2$, and (ii) $\destutter(u) = \sigma_1 \cdot \destutter(\sigma_2 v)$ otherwise.
By extension, for a set $L \subseteq \Sigma^*$ of finite words, we have $\destutter(L) = \{\destutter(u) \st u \in L\}$.

Moreover, given an integer $k \geq 0$, we define $\stutter_k : \Sigma^* \to \Sigma^*$ such that $\stutter_k(u) = \{v \in \Sigma^* \st |v| = k \land \destutter(v) = \destutter(u)\}$ if $k \geq |u|$, and $\stutter_k(u) = \emptyset$ otherwise.

\subsection{Signal Model}

\begin{definition}
	A \emph{signal} is a right-continuous, left-limited, non-Zeno, piecewise-constant function $x : [0,d) \to \R$ where $d \R_{> 0}$ is the duration of $x$.
\end{definition}

\TODO: right-cont, left-lim, nonzeno, piecewise-constant, interval operations, minkowski sum

Given an interval $I \subset \R_{\geq 0}$, a \emph{segmentation} of $I$ is a partition of $I$ into finitely many intervals $I_1, \ldots, I_k$ of the form $I_j = [t_j, t_{j+1})$ such that $t_j < t_{j+1}$ for all $1 \leq j \leq k$.
By extension, a segmentation of a collection of signals with the same temporal domain $I$ is a segmentation of $I$.

%Let $x : [0,d) \to \B$ be a signal.
%The \emph{piecewise segmentation} of $x$ is the smallest segmentation $I_1, \ldots, I_k$ of $[0,d)$ such that $x(t) = x(t')$ for every $1 \leq i \leq k$ and $t,t' \in I_i$.
%Then, the \emph{profile} of $x$ is the nonstuttering sequence of truth values encoding the behavior of $x$ with respect to its piecewise segmentation, i.e., $\prof(x) = \destutter((x(t_i))_{1 \leq i \leq k})$ where $I_i = [t_i, t_{i+1})$ for all $1 \leq i \leq k$.

We consider an asynchronous and loosely-coupled message-passing system of $N \geq 2$ reliable agents, denoted $A_1, \ldots, A_N$.
The agents produce a set of $n \geq 2$ signals $x_1, \ldots, x_n$ where $x_i : [0,d) \to \R$ for some $d \in \R_{> 0}$. \rednote{we take the longest signal and pad the rest?}
The function $\pi : \{x_1, \ldots, x_n\} \to \{A_1, \ldots, A_N\}$ maps each signal to the agent it is produced by.
The agents do not share memory or a global clock.
Only to formalize statements, we speak of a hypothetical global clock and denote its value by $T$.
For denoting local time values, we use the lowercase letters $t$ and $s$.

We represent the local clock of the agent $A_i$ as an increasing and divergent function $c_i : \R_{\geq 0} \to \R_{\geq 0}$ that maps a global time $T$ to a local time $c_i(T)$ of $A_i$.
We denote by $c_i^{-1}$ the inverse of the local clock function $c_i$.
We assume that the system is \emph{partially synchronous}: the agents use a clock synchronization algorithm that guarantees a bounded clock skew with respect to the global clock, i.e., $|c_i(T) - T| < \varepsilon$ for all $1 \leq i \leq N$ and $T \in \R_{\geq 0}$, where $\varepsilon \in \R_{> 0}$ is the maximum clock skew.

An \emph{event} of a given signal $x_i$ is a pair $(t, x_i(t))$ where $t$ is a local clock value.
We denote by $E_i$ the set of events of signal $x_i$.
An \emph{edge} of $x_i$ is an event $(t, x_i(t))$ such that $\lim_{s \to t^-} x_i(s) \neq \lim_{s \to t^+} x_i(s)$.
In particular, it is a \emph{rising} edge if $\lim_{s \to t^-} x_i(s) < \lim_{s \to t^+} x_i(s)$, and a \emph{falling} edge otherwise.
We respectively denote by $D_{x_i}^\uparrow$ and $D_{x_i}^\downarrow$ the sets of rising and falling edges of $x_i$, and we let $D_{x_i} = D_{x_i}^\uparrow \cup D_{x_i}^\downarrow$.
%Moreover, we let $D = \bigcup_{i = 1}^{n} D_{x_i}$ to denote the set of events of a distributed signal $S = (x_1, \ldots, x_n)$ 

We assume that signals have \emph{bounded variability} with respect to the global clock: for each signal $x_i$ and every pair $(t, x_i(t)), (t', x_i(t')) \in D_{x_i}$ of edges, we have $|c_j^{-1}(t) - c_j^{-1}(t')| \geq \delta$, where $A_j = \pi(x_i)$ is the agent that produces $x_i$ and $\delta \in \R_{\geq 0}$ is the bounded variability constant.

\begin{definition} \label{defn:hb}
	A \emph{distributed signal} is a pair $(S, {\hb})$ where $S = (x_1, \ldots, x_n)$ is a vector of signals and ${\hb}$ is the happened-before relation between events in signals extended with the partial synchrony assumption as follows.
	\begin{itemize}
		\item For every agent, the events of its signals are totally ordered, i.e., for all $1 \leq i,j \leq n$ with $\pi(x_i) = \pi(x_j)$ and all $(t, x_i(t)) \in E_i$ and $(t', x_j(t')) \in E_j$, if $t < t'$ then $(t, x_i(t)) \hb (t', x_j(t'))$.
		\item Every pair of events whose timestamps are more than $2 \varepsilon$ apart is totally ordered, i.e., for all $1 \leq i,j \leq n$ and all $(t, x_i(t)) \in E_i$ and $(t', x_j(t')) \in E_j$, if $t + 2\varepsilon < t'$ then $(t, x_i(t)) \hb (t', x_j(t'))$. %TODO: strict or nonstrict?
	\end{itemize}
\end{definition}

\begin{example}
	\TODO: distributed signal, happened-before relation
\end{example}

\begin{definition}
	Let $(S, {\hb})$ be a distributed signal of $n$ signals, and $E = \bigcup_{i = 1}^{n} E_i$ be the set of its events.
	A set $C \subseteq E$ is a \emph{consistent cut} iff for every event in $C$, all events that happened before  it also belong to $C$, i.e., for all $e, e' \in E$, if $e \in C$ and $e' \hb e$, then $e' \in C$.
\end{definition}

%todo: check below 2 paragraphs for consistency with using a reference clock
We denote by $\CC(T)$ the (infinite) set of consistent cuts at global time $T$.
Given a consistent cut $C$, its \emph{frontier} $\fr(C) \subseteq C$ is the set consisting of the last events in $C$ of each signal, i.e., $\fr(C) = \bigcup_{i = 1}^{n} \{ (t, x_i(t)) \in E_i \cap C \st \forall t' > t : (t', x_i(t')) \notin E_i \cap C \}$.

A \emph{consistent cut flow} is a function $f : I \to 2^E$, where $I \subset \R_{\geq 0}$ is a left-closed right-open interval, that maps a global clock value $T$ to the frontier of a consistent cut at time $T$, i.e., $f(T) \in \{\fr(C) \st C \in \CC(T)\}$.
Moreover, for all $T,T' \in I$ and $1 \leq i \leq n$, if $T < T'$, then for every pair of events $(c_j(T), x_i(c_j(T))) \in f(T)$ and $(c_j(T'), x_i(c_j(T'))) \in f(T')$ we have $(c_j(T), x_i(c_j(T))) \hb (c_j(T'), x_i(c_j(T')))$, where $A_j = \pi(x_i)$.
We denote by $\CCF(S,{\hb})$ the set of all consistent cut flows of the distributed signal $(S,{\hb})$.

\begin{example}
	\TODO: consistent cut, frontier, consistent cut flow
\end{example}


\subsection{Signal Temporal Logic} \label{sec:stl}
Let $\AP$ be a set of atomic propositions.
The syntax is given by the following grammar where $p \in \AP$ and $I \subseteq \R_{\geq 0}$ is an interval.
$$ \varphi :=  p ~|~ \lnot \varphi ~|~ \varphi \land \varphi ~|~ \varphi \until_I \varphi$$

A trace $w = (x_1, \ldots, x_n)$ is a vector of synchronous signals $x_i : [0,d) \to \R$.
We express atomic propositions as functions of trace values at a time point $t$, i.e., a proposition $p \in \AP$ is defined as $f_p(x_1(t), \ldots, x_n(t)) > 0$ where $f_p : \R^n \to \R$ is a function.

Below we recall the infinite-trace qualitative semantics of STL defined over $\B$.
Let $w = (x_1, \ldots, x_n)$ be a trace with $x_i : [0,d) \to \R$ for all $1 \leq i \leq n$, and $\varphi_1, \varphi_2$ be STL formulas.
We assume that $t,t',t'' \in [0,d)$.
\begin{align*}
	[w,t \models p] &\iff f_p(x_1(t), \ldots, x_n(t)) > 0 \\
	[w,t \models \lnot \varphi] &\iff \overline{[w,t \models \varphi]} \\
	[w,t \models \varphi_1 \land \varphi_2] &\iff [w,t \models \varphi_1] \land [w,t \models \varphi_1] \\
	[w,t \models \varphi_1 \until \varphi_2] &\iff \exists t' \in t \oplus I : [w,t' \models \varphi_2] \land \forall t'' \in (t, t') : [w,t'' \models \varphi_1] \\
%	[w,t \models \LTLf_I \varphi_1] &\iff \exists t' \in t \oplus I : [w,t' \models \varphi_1] \\
%	[w,t \models \LTLg_I \varphi_1] &\iff \forall t' \in t \oplus I : [w,t' \models \varphi_1] \\
\end{align*}

We additionally use the following standard abbreviations: 
$\true = p \lor \lnot p$,
$\false = \lnot \true$,
$ \varphi_1 \lor \varphi_2 = \lnot \varphi_1 \land \lnot \varphi_2$,
$\LTLf_I \varphi = \true \until_I \varphi$, and
$\LTLg_I \varphi = \lnot \LTLf_I \lnot \varphi$.
Moreover, the untimed temporal operators are defined through their timed counterparts on the interval $(0,\infty)$, e.g., $\varphi_1 \until \varphi_2 = \varphi_1 \until_{(0,\infty)} \varphi_2$.
Note that our interpretation of the untimed until operator is strict.
The non-strict variant can be defined in terms of the strict one as follows: $\varphi_1 \underline{\until} \varphi_2 = \varphi_2 \lor (\varphi_1 \land (\varphi_1 \until \varphi_2))$.
Finally, we simply write $[w \models \varphi]$ for $[w,0 \models \varphi]$.

The semantics above is defined for infinite traces while a distributed signal has finite length.
To bridge this gap, we take the standard extension to a 3-valued semantics:
Given a finite-length synchronous trace $w$ and an STL formula $\varphi$, 
we define $[w \models \varphi]_3 = \top$ iff $[w w' \models \varphi]$ for every infinite-length signal $w'$, define $[w \models \varphi]_3 = \bot$ iff $[w w' \models \lnot \varphi]$ for every infinite-length signal $w'$, and $[w \models \varphi]_3 = \?$ otherwise.