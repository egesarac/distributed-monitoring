\section{Our Approach}
We assume a central monitor that knows the values of $\delta$ and $\varepsilon$ as well as the initial values of the signals.

\subsection{Uncertainty Regions and Canonical Segmentations} \label{sec:segment}

Consider a signal $x : [0,d) \to \R$ with a single rising edge from value 2 to 3 at local time $t$.
The monitoring observing this signal needs to take into account the clock skew: according to the global clock, the rising edge of $x$ occurs in the range $(t - \varepsilon, t + \varepsilon)$.
This range is called an \emph{uncertainty region} because the monitor cannot tell the value of $x$ precisely, but only that it changes from 2 to 3.
To systematically reason about uncertainty regions of signals, we use the notion of segmentation of temporal domains of signals.

Let $x : [0,d) \to \R$ be a signal and $D_x = \{(t_1, x(t_1)), \ldots, (t_m, x(t_m))\}$ be the set of edges of $x$, given in an increasing order of local clock values.
For $1 \leq i \leq m$, let $\theta_{\text{lo}}(x,t_i) = \max\{0, \max_{j \in \{1, i\}} t_j - \varepsilon - (j-i)\delta\}$ and $\theta_{\text{hi}}(x,t_i) = \min_{i \leq j \leq m} t_j + \varepsilon - (j-i)\delta$.
Intuitively, $\theta_{\text{lo}}$ and $\theta_{\text{hi}}$ give us the lower and upper bounds on the value of the global clock for a given edge.
We use these to describe a canonical segmentation of a distributed signal.

Let $(S,{\hb})$ be a distributed signal of $n$ signals.
For each signal $x_i$, let $F_i = \{\theta_{\text{lo}}(x_i, t_j) \st (t_j, x_i(t_j)) \in D_{x_i}\} \cup \{\theta_{\text{hi}}(x_i, t_j) \st (t_j, x_i(t_j)) \in D_{x_i}\}$ be the set of bounds on the global clock values for its edges.
Let $d' = \max(d, \max (\bigcup_{i = 1}^{n} F_i))$. %\rednote{we assume padding of the `retimed' signals too?}
We define $F = \{0, d'\} \cup \bigcup_{i = 1}^{n} F_i$ and let $(a_j)_{1 \leq j \leq |F|}$ be an increasing sequence of clock values corresponding to the elements of $F$.
We define the \emph{canonical segmentation} of $(S,{\hb})$ as $G_S = \{I_1, \ldots, I_{|F| - 1}\}$ where $I_j = [a_j, a_{j+1})$ for all $1 \leq j < |F|$.
%Moreover, given $t \in [0,d)$, we define $\theta(S,t) = I'$ where $I' \in G_S$ and $t \in I'$.

\begin{example}
	\TODO: canonical segmentation
\end{example}

\subsection{Value Expressions}
In uncertainty regions, although the monitor is not able to determine the exact value of a signal at a time point, it knows how the signal values change.
To capture this information, we use short sequences of possible signal values capturing how a signal behaves within a segment and several operations on them to propagate the information from real-valued signals to truth values subformulas.

%In the next section, we will map each segment of a signal to a set of value expressions corresponding to its set of possible behaviors in this time segment.
%Moreover, we propagate this information from real-valued signals to truth values of atomic propositions and subformulas to complete the evaluation of a distributed signal and a formula.

Let $\Sigma$ be a finite alphabet corresponding to a set of signal values.
A \emph{value expression} is a finite word in $\destutter(\Sigma^*)$.
Given two value expressions $u_1$ and $u_2$, we define their asynchronous product as $u_1 \otimes u_2 = \{ (v_1,v_2) \st \max(|u_1|, |u_2|) \leq k \leq |u_1| + |u_2|, v_1 \in \stutter_k(u_1), v_2 \in \stutter_k(u_2) \}$.
By extension, given $L_1, L_2 \subseteq \destutter(\Sigma^*$), we have $L_1 \otimes L_2 = \{ u_1 \otimes u_2 \st u_1 \in L_1, u_2 \in L_2 \}$.
For a function $f : \Sigma^2 \to \R$, we refer to the set of value expressions that can be obtained by applying $f$ to the potential ``interleavings'' of $u_1$ and $u_2$ as follows: $f(u_1, u_2) = \destutter(\{(f(v_1[i], v_2[i]))_{0 \leq i < |v_1|} \st (v_1, v_2) \in u_1 \otimes u_2\})$.
Moreover, for $c \in \R$, we write $f(u_1, u_2) \geq c$ to denote the set $\destutter(\{(f(v_1[i], v_2[i]) \geq c)_{0 \leq i < |v_1|} \st (v_1, v_2) \in u_1 \otimes u_2\})$ where $f(v_1[i], v_2[i]) \geq c$ is 1 if the inequality holds, and it is 0 otherwise.

Note that the definitions above can be extended to handle more than two signals.
As we will detail in the next section, after computing the value expressions of real-valued signals, we compute those of satisfaction signals of atomic propositions.
We use these boolean value expressions in the evaluation of the STL formula.
Therefore, we assume a binary alphabet in the sequel.

Let $\Sigma = \{0,1\}$.
Given $u,v \in \Sigma^*$ of the same length, we respectively denote by $u \BitAnd v$ the `bitwise and' operation, and by $\BitNeg u$ the `bitwise negation' operation.
In addition, given nonempty words $u,v \in \Sigma^*$ of length $\ell$, we define auxiliary bitwise temporal operators as follows:
\begin{align*}
\mathsf{E} u &= \left( \max_{i \leq j < \ell} u[j] \right)_{0 \leq i < \ell} \\
\mathsf{A} u &= \left( \min_{i \leq j < \ell} u[j] \right)_{0 \leq i < \ell} \\
u \mathsf{U}^0 v &= \left( \max_{i \leq j < \ell} \left( \min \left( v[j], \min_{i \leq k < j} u[k] \right) \right) \right)_{0 \leq i < \ell} \\
u \mathsf{U}^1 v &= \left( \max \left( u[i..] \mathsf{U}^0 v[i..], \mathsf{A} u[i..] \right) \right)_{0 \leq i < \ell} \\
\end{align*}
Observe that $\mathsf{A} u = u \mathsf{U}^1 0^\ell$ and $\mathsf{E} u = 1^\ell \mathsf{U}^0 u$.

Below we use finite words from $\destutter(\Sigma^*)$ to denote how the values of a satisfaction signal evolve through intervals that include uncertainty and interleavings.
We use the following grammar to encode such expressions.
$$ \psi := 0 ~|~ 1 ~|~ \lnot \psi ~|~ \psi \sqcap \psi  ~|~ \psi \cdot \psi ~|~ \psi + \psi ~|~ \psi \until^0 \psi ~|~ \psi \until^1 \psi $$ %  ~|~ \LTLf \psi ~|~ \LTLg \psi $$  %~|~ \psi \triangleleft \psi ~|~ \psi \triangleright \psi ~|~ \psi \diamond \psi

We denote by $\Psi$ the set of all expressions generated by the above grammar.
Letting $\psi_1, \psi_2 \in \Psi$, we inductively define the semantics for our encoding.
\begin{align*}
		\llbracket 0 \rrbracket &=  \{ 0 \} \\
		\llbracket 1 \rrbracket &=  \{ 1 \} \\
		\llbracket \lnot \psi_1 \rrbracket &= \{ \BitNeg u \st u \in \llbracket \psi_1 \rrbracket \} \\
		\llbracket \psi_1 \sqcap \psi_2 \rrbracket &= \destutter(\{ v_1 \BitAnd v_2 \st (v_1,v_2) \in \llbracket \psi_1 \rrbracket \otimes \llbracket \psi_2 \rrbracket \}) \\
		\llbracket \psi_1 \cdot \psi_2 \rrbracket &= \destutter(\{ u_1 u_2 \st u_1 \in \llbracket \psi_1 \rrbracket, u_2 \in \llbracket \psi_2 \rrbracket \}) \\
		\llbracket \psi_1 + \psi_2 \rrbracket &= \llbracket \psi_1 \rrbracket \cup \llbracket \psi_2 \rrbracket \\
		\llbracket \psi_1 \until^0 \psi_2 \rrbracket &= \destutter(\{ v_1 \mathsf{U}^0 v_2  \st (v_1,v_2) \in \llbracket \psi_1 \rrbracket \otimes \llbracket \psi_2 \rrbracket \}) \\
		\llbracket \psi_1 \until^1 \psi_2 \rrbracket &= \destutter(\{ v_1 \mathsf{U}^1 v_2  \st (v_1,v_2) \in \llbracket \psi_1 \rrbracket \otimes \llbracket \psi_2 \rrbracket \}) \\
%		\llbracket \LTLf \psi \rrbracket &= \destutter(\{ \mathsf{E} u \st u \in \llbracket \psi \rrbracket \}) \\
%		\llbracket \LTLg \psi \rrbracket &= \destutter(\{ \mathsf{A} u \st u \in \llbracket \psi \rrbracket \}) \\
\end{align*}

\TODO: give intuition for each operation

We additionally define $\psi_1 \sqcup \psi_2 = \lnot (\lnot \psi_1 \sqcap \lnot \psi_2)$, $\LTLf \psi = 1 \until^0 \psi$, and $\LTLg \psi = \psi \until^1 0$, and 
Moreover, one can easily verify that $\destutter(\Sigma^*)$ is closed under these operations.

%\TODO: explain how we represent the value expressions and how we do the operations (pref, suff, conj, neg. and others?)

%In order to implement these efficiently, we represent subsets of $\destutter(\Sigma^*)$ as follows.
%2D ARRAY STUFF.
%For example, ...
%PREFIX, SUFFIX, NEGATION, AND.






%Let $w = (\psi_1, I_1), (\psi_2, I_2), \ldots , (\psi_n, I_n)$ a continuous-time finite-length boolean signal with uncertainty such that $I_i = [t_i, t_{i+1})$, where $t_1 = 0$ and for all $i \in [0,n)$, $t_{i+1} > t_i$.
%In every interval $I_i$, the signal can assume value patterns defined by the qualitative word $v \in \llbracket \psi_i \rrbracket$.
%
%\begin{example}
%	A signal $w$ of duration $d$ with a single rising edge at time $t$ is represented as $(0, [0, t)), (1, [t, d))$.
%	This same signal, interpreted at some other process, assuming a maximal clock skew $0 < \varepsilon < t$ is represented as $(0, [0, t - \varepsilon)), (0\cdot 1, [t - \varepsilon, t + \varepsilon)), (1, [t + \varepsilon, d))$. %todo: explain the idea (observing from different process etc) at the beginning of the section
%\end{example}
%
%%todo: rephrase, show closure under pfx and sfx, show closure under conjunction
%The first two operations on $\psi$ that we need are its prefix $\pfx(\psi)$ and suffix $\sfx(\psi)$:
%\begin{equation}
%	\begin{array}{lcl}
%		\pfx(\psi) & = & \{ w~|~ \exists w' \text{ s.t. } w\cdot w' \in \llbracket \psi \rrbracket \} \\
%		\sfx(\psi) & = & \{ w~|~ \exists w' \text{ s.t. } w'\cdot w \in \llbracket \psi \rrbracket \} \\
%	\end{array}
%\end{equation}
%
%\begin{example}
%	We need the prefix and the suffix operations when we need to split an interval into two intervals and describe what is allowed in each interval. For example, consider the interval $(0\cdot 1, [t_1, t_2))$ from the previous example and a time $t_1 < t' < t_2$ at which we want to split the interval. We now need to describe what are the qualitative patterns that we can observe in $[t_1, t)$ and in $[t', t_2)$. It is not hard to see that in $[t_1, t')$, we can see either $0$ or $0 \cdot 1$, while in $[t_1, t')$ we can see either $0 \cdot 1$ or $1$. It follows that we can represent $(0 \cdot 1, [t_1, t2)$ as $(\pfx(0\cdot 1), [t_1, t')), (\sfx(0 \cdot 1), [t', t_2))$, where $\pfx(0 \cdot 1) = 0 \triangleleft 1$ and $\sfx(0 \cdot 1) = 0 \triangleright 1$. We want to have simple, syntactic and inductive definition of the prefix and suffix operations.
%\end{example}
%
%%todo: define a function from signals and segments to truth expressions

\subsection{The Overapproximation Logic}
Given a distributed signal $(S,{\hb})$, our logic maps a segmentation of $(S,{\hb})$ to a set of value expressions.
Using these, first we compute the value expressions for atomic propositions, and then inductively evaluate the subformulas.

We assume that, at any segment, the signal can realize any value expression of the segment.
This leads to an overapproximation of the set of possible behaviors as demonstrated below.

\begin{example}
	\TODO: how value expressions overapproximate possible behaviors
\end{example}

Let $(S,{\hb})$ be a distributed signal of $n$ signals $x_1, \ldots x_n$, let $G_S$ be its canonical segmentation, and let $\varphi$ be the formula to be evaluated.
%For a given STL formula $\varphi$, let $B_\varphi$ be the set of finite and positive bounds that occur in the timed operators of $\varphi$.
%Recall the set $F$ we used in \Cref{sec:segment} to define the canonical segmentation of $(S,{\hb})$.
%To take into account the timed operators in $\varphi$, we extend $F$ as follows:
%$F_\varphi = F \cup \{t - b \st t \in F, b \in B_\varphi, t > b\}$.
%Now, let $d_1, \ldots, d_M$ be the elements of $F_\varphi$ in increasing order.
%We extend the canonical segmentation $G_S$ as follows:
%$G_S^\varphi = \{I_1', \ldots I_{M-1}'\}$ where $I_i' = [d_i, d_{i+1})$.
We want to compute the value expressions of each signal in every interval in the canonical segmentation $G_S$.
For this, we first let $\gamma(I, i) = \{\epsilon\}$ for all $I \in G_S$ and $1 \leq i \leq n$, then incrementally add value expressions to $\gamma(I, i)$ to capture the potential behaviors of $x_i$ in the interval $I$.

For each $x_i$, consider the set $X_i = \cup_{1 \leq j < |D_i|} \{K_{i,j}\}$ of its uncertainty regions, i.e.,  $K_{i,j} = [l_j, h_j)$ where $l_j = \theta_{\text{lo}}(x_i, t_j)$ and $h_j = \theta_{\text{hi}}(x_i, t_j)$.
These intervals have the value expression $v_{i,j} = \nu_1 \nu_2$ if $\lim_{s \to t_j^-} x_i(s) = \nu_1$ and $\lim_{s \to t_j^+} x_i(s) = \nu_2$.
%We denote the value expression of interval $K_{i,j}$ by $v_{i,j}$.
Note that the value expressions of the remaining intervals are constant.

Let $G_S = \{I_1, \ldots, I_{n'}\}$ where $I_\ell = [a_\ell, a_{\ell+1})$ for all $1 \leq \ell \leq n'$.
For each $1 \leq i \leq n$ and $1 \leq k \leq n'$ let $Y_{k,i} = \{ K_{i,j} \in X_i \st I_k \subseteq K_{i,j}\}$ be the uncertainty regions of $x_i$ that contain the interval $I_k$.
%For each interval $I \in G_S$ where $I = [l,h)$ and each $1 \leq i \leq n$, let $Y_{I,i} = \{ K_{i,j} \in X_i \st I \subseteq K_{i,j}\}$ be the uncertainty regions of $x_i$ that contain the interval $I$.
We compute the value expressions of signals for the canonical segmentation as follows.
\begin{itemize}
	\item For each $1 \leq i \leq n$ in increasing order, for each $1 \leq k \leq n'$ in increasing order, and for each $K_{i,j} = [l_j, h_j) \in Y_{k,i}$ in increasing order, do the following:
	\begin{itemize}
		\item If $a_k = l_j$ and $a_{k+1} = h_j$, then $\gamma(I_k, i) = \gamma(I_k, i) \cdot \{v_{i,j}\}$.
		\item If $a_k = l_j$ and $a_{k+1}  < h_j$, then $\gamma(I_k, i) = \gamma(I_k, i) \cdot \pfx(v_{i,j})$.
		\item If $a_k > l_j$ and $a_{k+1}  = h_j$, then $\gamma(I_k, i) = \gamma(I_k, i) \cdot \sfx(v_{i,j})$.
		\item If $a_k > l_j$ and $a_{k+1}  < h_j$, then $\gamma(I_k, i) = \gamma(I_k, i) \cdot \infx(v_{i,j})$.
	\end{itemize}
\end{itemize}

%\begin{itemize}
%	\item If $h < h_j$, then add $\pfx(v_{i,j})$ to $\gamma(I,i)$.
%	\item If $l > l_j$ then add $\sfx(v_{i,j})$ to $\gamma(I,i)$.
%	\item If $l = l_j$ and $h = h_j$, then add $v_{i,j}$ to $\gamma(I, i)$.
%\end{itemize}

Given a set $L \subseteq \Sigma^*$, we define $\first(L) = \{ u[0] \st u \in L\}$.
Let $\tau_S : [0,d) \to G_S$ be a function that maps each time stamp to the interval of the canonical segmentation that it belongs to, i.e., $\tau_S(t) = [t', t'')$ where $t' \leq t < t''$ and $[t', t'') \in G_S$.
%For $t_1, t_2 \in [0,d)$, if $\tau_S(t_1) \neq \tau_S(t_2)$, let $(J_i)_{1 \leq i \leq k}$ the (potentially empty) sequence of intervals that are strictly between $\tau_S(t_1)$ and $\tau_S(t_2)$.

Let $(S, {\hb})$ be a distributed signal of $n$ signals, $t \in [0,d)$ be a time value, $I \subseteq \R_{\geq 0}$ be a time interval with the end points $a < b$.
Note that timed until can be expressed using timed eventually, timed globally, and untimed until~\cite{DBLP:journals/sttt/MalerN13}.
We provide the semantics of our logic based on this remark. %and the STL syntax given in \Cref{sec:stl}.

Let $I \subseteq [0,d)$ be an interval with the end-points $a < b$.
Below we define the \emph{profile} of $I \subseteq [0,d)$ with respect to $S$ and $\varphi$.
Given $t \in [0,d)$, we let $\mathsf{end}(t, S)$ be true iff $t=d$ or $\tau_S(t) = [t, t')$ for some $t' \in [0,d)$.
Let $(H_i)_{1 \leq i \leq K}$ be a (possibly empty) sequence of intervals where $H_i = [d_i, d_{i+1}) \in G_S$ and $a < d_i < d_{i+1} < b$ for all $1 \leq i \leq K$ and $K$ is the number of intervals in $G_S$ whose endpoints are strictly between $a$ and $b$.
We denote by $Q(S, a, b)$ the concatenation of the value expressions of the intervals $H_1, \ldots, H_K$, i.e., $Q(S, a, b) = [S, d_1 \models \varphi] \cdot \ldots \cdot [S, d_K \models \varphi]$.
%\rednote{fix the cases: $a \in I$? $b \in I$?}

\scriptsize
%\scalebox{0.8}{
\begin{equation*} 
	\mathsf{profile}(I, S, \varphi) =
	\begin{cases}
%		[S, a \models \varphi] &\text{if $\tau_S(a) = \tau_S(b) \land \mathsf{start}(a, S) \land \mathsf{end}(b, S)$} \\
		\pfx([S, a \models \varphi]) &\text{if $\tau_S(a) = \tau_S(b) \land \mathsf{end}(a, S)$} \\
%		\sfx([S, a \models \varphi]) &\text{if $\tau_S(a) = \tau_S(b) \land \lnot\mathsf{start}(a, S) \land \mathsf{end}(b, S)$} \\
		\infx([S, a \models \varphi]) &\text{if $\tau_S(a) = \tau_S(b) \land \lnot\mathsf{end}(a, S)$} \\
		[S, a \models \varphi] \cdot Q(S, a, b) \cdot \first([S, b \models \varphi]) &\text{if $\tau_S(a) \neq \tau_S(b) \land \mathsf{end}(a, S) \land \mathsf{end}(b, S) \land b \in I$} \\
		[S, a \models \varphi] \cdot Q(S, a, b) &\text{if $\tau_S(a) \neq \tau_S(b) \land \mathsf{end}(a, S) \land \mathsf{end}(b, S) \land b \notin I$} \\
		[S, a \models \varphi] \cdot Q(S, a, b) \cdot \pfx([S, a \models \varphi]) &\text{if $\tau_S(a) \neq \tau_S(b) \land \mathsf{end}(a, S) \land \lnot\mathsf{end}(b, S)$} \\
		\sfx([S, a \models \varphi]) \cdot Q(S, a, b) \cdot \first([S, b \models \varphi]) &\text{if $\tau_S(a) \neq \tau_S(b) \land \lnot\mathsf{end}(a, S) \land \mathsf{end}(b, S) \land b \in I$} \\
		\sfx([S, a \models \varphi]) \cdot Q(S, a, b) &\text{if $\tau_S(a) \neq \tau_S(b) \land \lnot\mathsf{end}(a, S) \land \mathsf{end}(b, S) \land b \notin I$} \\
		\sfx([S, a \models \varphi]) \cdot Q(S, a, b) \cdot \pfx([S, b \models \varphi]) &\text{if $\tau_S(a) \neq \tau_S(b) \land \lnot\mathsf{end}(a, S) \land \lnot\mathsf{end}(b, S)$} 
	\end{cases}
\end{equation*}
%}
\normalsize

%Given an interval $J \subseteq [0,d)$, let $\mathsf{Int}(d, J)$ be the set of all intervals with the same length and type as $J$.
%Let $\mathsf{PInt}(d, J, S, \varphi)$ be the set of profiles of all the intervals in $\mathsf{Int}(d, J)$ with respect to $S$ and $\varphi$.

For $s_1, s_2 \in [0,d)$, let $\mathsf{profsbw}(I, S, \varphi, s_1, s_2) = \{\textit{Pr}_1, \ldots, \textit{Pr}_M\}$ be the set of profiles with respect to $S$ and $\varphi$ of intervals in $G_S$ that are strictly between $s_1$ and $s_2$ and are of the same length and type as $I$, given in an increasing order of the starting points of the intervals they correspond to. \rednote{define same type}
Let $\tau_S(t) = [t', t'')$.
We are now ready to define the semantics.

\scriptsize
\begin{align*}
	[S,t \models p] &=  f_p(\gamma(\tau_S(t), 1), \ldots, \gamma(\tau_S(t), n)) > 0 \\
	[S,t \models \lnot \varphi] &= \lnot [S,t \models \varphi] \\
	[S,t \models \varphi_1 \land \varphi_2] &= [S,t \models \varphi_1] \sqcap [S,t \models \varphi_2] \\
	[S,t \models \varphi_1 \until \varphi_2] &= \destutter(\{ [S,t \models \varphi_1] \until^b [S,t \models \varphi_2] \st \text{ if $t'' < d$ then $b \in \first([S,t'' \models \varphi_1 \until \varphi_2])$, otherwise $b = 0$} \}) \\
	[S,t \models \LTLf_I \varphi] &= \destutter(\{ \mathsf{E}(\textit{Pr}_1 \cdot \ldots \cdot \mathsf{E}(\textit{Pr}_M) \}) \text{ where $\textit{Pr}_i \in \mathsf{profsbw}(I, S, \varphi, t' + a, t'' + b)$} \\
	[S,t \models \LTLg_I \varphi] &= \destutter(\{ \mathsf{A}(\textit{Pr}_1 \cdot \ldots \cdot \mathsf{A}(\textit{Pr}_M) \}) \text{ where $\textit{Pr}_i \in \mathsf{profsbw}(I, S, \varphi, t' + a, t'' + b)$}
\end{align*}
\normalsize




