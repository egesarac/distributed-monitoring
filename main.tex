\documentclass[envcountsame, runningheads]{llncs}

\usepackage[utf8]{inputenc}
\usepackage[T1]{fontenc}
\usepackage{lmodern}
\usepackage[hidelinks]{hyperref}
\usepackage{diagbox}

\usepackage{amssymb}
\usepackage{amsmath}
\usepackage{tikz}
\usetikzlibrary{backgrounds, positioning, arrows, calc}
\usepackage{mathtools}
\usepackage[linesnumbered,vlined,noend]{algorithm2e}
\usepackage[inline]{enumitem}  

% ----- DEBUG
\newcommand{\noteitem}{\ensuremath{\;{\bullet}\;}}
\setlength{\marginparwidth}{4.2cm}
\setlength{\marginparsep}{0.3cm}
\usepackage{todonotes}
\newcommand{\answer}[1]{{\color{white}#1}}
\newcommand{\orangenote}[2][]{{\todo[color=orange!80,size=\footnotesize,#1]{\normalcolor\normalfont#2}}}
\newcommand{\bluenote}[2][]{{\todo[color=cyan!80,size=\footnotesize,#1]{\normalcolor\normalfont#2}}}
\newcommand{\greennote}[2][]{{\todo[color=green!80!black,size=\footnotesize,#1]{\normalcolor\normalfont#2}}}
\newcommand{\rednote}[2][]{{\todo[color=magenta!80,size=\footnotesize,#1]{\normalcolor\normalfont#2}}}
\newcommand\alert[1]{\textcolor{red}{#1}}
\newcommand\TODO{\textcolor{red}{TODO}}

% ----- FIGURES
\tikzstyle{state}=[thick,minimum size=18pt, circle,draw]
\tikzstyle{transition}=[->,thick,>=stealth,shorten >=1pt,shorten <=1pt]
\tikzstyle{final}=[after node path={ node[state, scale=.8] at (\tikzlastnode) {} }]
\tikzstyle{initial}=[after node path={
	[to path={[transition] (\tikztostart) -- (\tikztotarget)}]
	(\tikzlastnode)++(180:22pt) edge (\tikzlastnode)
}]
\tikzset{
	bg/.default={},
	bg/.style={execute at end picture={
			\begin{scope}[on background layer]
				\node[xshift=-1mm, yshift=-1mm] (sw) at (current bounding box.south west) {};
				\node[xshift=1mm, yshift=1mm] (ne) at (current bounding box.north east) {};
				\node[xshift=1mm, yshift=-1mm] (nw) at (current bounding box.north west) {};
				\fill[fill=black!10,rounded corners] (sw) rectangle (ne);
				
				\ifx&#1&\else
				\node[anchor=north east, xshift=2pt] at (nw) {#1};
				\fi
			\end{scope}
	}},
}

% ----- NUMERICAL SETS
\newcommand{\Z}{\mathbb{Z}}
\newcommand{\ZZ}{\overline{\ref{ex:minresp}Z}}
\newcommand{\N}{\mathbb{N}}
\newcommand{\NN}{\overline{\N}}
\newcommand{\R}{\mathbb{R}}
\newcommand{\RR}{\R^{\pm \infty}}
\newcommand{\Q}{\mathbb{Q}}
\newcommand{\D}{\mathbb{D}}
\newcommand{\B}{\mathbb{B}}

% ------ FUNCTIONS
\newcommand{\avg}{\text{\normalfont avg}}
\newcommand{\disc}{\text{\normalfont disc}}
\newcommand{\fin}{\text{\normalfont fin}} 
\newcommand{\limavg}{\text{\normalfont limavg}} 

% ----- LTL SYNTAX
\newcommand{\LTLf}{\ensuremath{\lozenge}}
\let\LTLfinally\LTLf
\let\LTLeventualy\LTLf
\newcommand{\LTLg}{\ensuremath{\square}}
\let\LTLalways\LTLg
\let\LTLglobally\LTLg
\newcommand{\LTLo}{\ensuremath{\bigcirc}}
\let\LTLnext\LTLo
\def\until{\,\mathcal{U}\,}
\def\since{\,\mathcal{S}\,}

\newcommand{\?}{\text{?}}

\newcommand{\req}{\texttt{rq}}
\newcommand{\ping}{\texttt{ping}}
\newcommand{\ack}{\texttt{ack}}
\newcommand{\gra}{\texttt{gr}}
\newcommand{\tick}{\texttt{tk}}
\newcommand{\other}{\texttt{oo}}
\newcommand{\off}{\texttt{off}}
%\newcommand{\ready}{\texttt{ready}}

% ----- MONITOR
\newcommand{\res}{\mathbf{r}}
\newcommand{\RES}{\mathbf{R}}
\newcommand{\ggeq}{\mathrel{\underline{\gg}}}
\newcommand{\lleq}{\mathrel{\underline{\ll}}}
\newcommand{\calM}{\mathcal{M}}
\newcommand{\calA}{\mathcal{A}}
\newcommand{\calB}{\mathcal{B}}
\newcommand{\prefixeq}{\preceq}
\newcommand{\prefix}{\prec}
\newcommand{\infPhi}{\inf\nolimits_\varPhi}
\newcommand{\supPhi}{\sup\nolimits_\varPhi}
\newcommand{\spec}{\textit{spec}}
\newcommand{\trace}{\textit{trace}}
\newcommand{\pref}{\textit{pref}}

% ----- GENERAL
\newcommand{\dom}[1]{{\textsf{dom}(#1)}}
\newcommand{\sem}[1]{{[\!\![#1]\!\!]}}
\newcommand{\suchthat}{\;\ifnum\currentgrouptype=16 \middle\fi|\;}
\let\st\suchthat
\newcommand{\defeq}{\coloneqq}
\DeclarePairedDelimiter\floor{\lfloor}{\rfloor}
\newcommand\Circle[1][1.4]{\tikz[baseline=-3.3]{\draw(0,0)circle[radius=#1mm];}}

% ----- THEOREMS
\newtheorem{observation}[theorem]{Fact}

\title{} 
\author{} %
\institute{} %
\authorrunning{} %
\date{}

%\renewcommand{\baselinestretch}{.98}

\begin{document}
	\maketitle

	\subsubsection*{Assumptions}
	\begin{enumerate}
		\item Every atomic proposition is uniquely mapped to a process, which transmit the changes in their truth values to a central monitor. The initial values of each atomic proposition is known by the monitor.
		\item There exists $\varepsilon > 0$ such that the local clocks of every pair of processes are guaranteed to be less than $\varepsilon$ apart at any given point. The value of the least upper bound on $\varepsilon$ is known by the monitor. 
		\item The atomic propositions have bounded variability, i.e., their values stay the same at least $\delta > 0$ time units after each change (with respect to an imaginary global clock). The value of the greatest lower bound on $\delta$ is known by the monitor. 
		\item The processes do not fail, the communication channels are FIFO and lossless.
		\item For online monitoring: The events may arrive out of order to the monitor. \rednote{Not taken into account yet.}
	\end{enumerate}
	
	\subsection*{LTL5 for Distributed Monitoring under Partial Synchrony}
	Let $\mathcal{P} = \{P_1, \ldots, P_m\}$ be a set of $m \geq 2$ processes and $\mathbb{P} = \{p_1, \ldots, p_n\}$ be a set of atomic propositions.
	The function $\pi : \mathbb{P} \to \mathcal{P}$ maps each proposition to the process it belongs to.
	Let $\B = \{\bot, \top\}$ be the boolean value domain, and $d \in \R_{\geq 0} \cup \{\infty\}$.
	A boolean signal $w : \mathbb{P} \times [0,d) \to \B$ maps each atomic proposition and time value to a boolean truth value.

	Note that every boolean signal $w$ can be represented in a discrete manner by identifying all the rising and falling edges.
	Let $\tau(w)$ be the discrete representation of a given signal $w$.
	Given an LTL formula $\psi$, we denote by $\mathcal{L}(\psi)$ the language of $\psi$ which consists of all infinite-length signals $w$ whose discrete representation $\tau(w)$ satisfies $\psi$.
	
	We define a 5-valued logic LTL5 to account for the clock skew.
	The truth domain of LTL5 consists of 5 values with $\bot < \bot_c < \? < \top_c < \top$, where $\overline{\bot} = \top$, $\overline{\bot_c} = \top_c$, and $\overline{\?} = \?$.
	To define LTL5, we first incorporate the value $\?$ in the standard logics FLTL and LTL3.
	
	\subsubsection*{Syntax}
	Defined by
	$ \varphi := \texttt{true} \,|\, p \,|\, \lnot \varphi \,|\, \varphi \lor \varphi \,|\, \varphi \until \varphi $
	where $p \in \mathbb{P}$.
	
	\subsubsection*{Semantics}
	Let $w : \mathbb{P} \times [0,d) \to \B$ be a boolean signal of finite length, $t \in [0,d)$ be a time value, $\varepsilon$ be the least upper bound on the clock skew, $\pi : \mathbb{P} \to \mathcal{P}$ a mapping from propositions to processes, $p \in \mathbb{P}$ be a proposition, and $P \in \mathcal{P}$ be a process.
	
	\greennote{Update: Add the 2nd preprocessing step into the semantics by defining two functions $\rho_{\text{lo}}$ and $\rho_{\text{hi}}$ that maps an event to its lower and upper bounds of uncertainty.}
	
	Given $w : \mathbb{P} \times [0,d) \to \B$ and $p \in \mathbb{P}$, let $S_{w,p} = \{t^p_1, \ldots, t^p_{n_p}\}$ be the set of timestamps for the rising/falling edges of $w$ for $p$.
	We define the (partial) functions $\rho_{\text{lo}}$ and $\rho_{\text{hi}}$ as follows, where $w$ is a signal, $p$ is a proposition, and $t^p_i \in S_{w,p}$ is a timestamp corresponding to an event.
	\rednote{TODO: Refine the edge cases below.}
	
	$$ \rho_{\text{lo}}(w, p, t^p_i) = \max_{j \in \{1, i\}} t^p_j - \varepsilon + (i-j)\delta $$
	
	$$ \rho_{\text{hi}}(w, p, t^p_i) = \min_{i \leq j \leq k} t^p_j + \varepsilon - (j-i)\delta $$
	where $i \leq k \leq n_p$ satisfies $t^p_{j + 1} - t^p_j < \delta + 2\varepsilon$ for all $i \leq j \leq k - 1$. \greennote{$\rho_{\text{hi}}(w, p, t^p_{j+1}) - (t^p_j + \varepsilon) < \delta$ \\ this would be useful in the offline case}
	
	\vspace{1em}
	FLTL:
	\begin{align*}
		[w,t \models \texttt{true}]_P &= \top \\
		[w,t \models p]_P &= \begin{cases}
			? &\text{if } \pi(p) \not = P \land \exists t' \in \ell_{w,p} : t \in (\rho_{\text{lo}}(w, p, t'), \rho_{\text{hi}}(w, p, t')) \\
			w(p,t) &\text{otherwise}			
		\end{cases} \\
		[w,t \models \lnot \varphi]_P &= \overline{[w,t \models \varphi]_P}\\
		[w,t \models \varphi_1 \lor \varphi_2]_P &= \max([w,t \models \varphi_1]_P, [w,t \models \varphi_2]_P)\\
		[w,t \models \varphi_1 \until \varphi_2]_P &= \max_{t \leq t'} \left( \min \left( [w,t' \models \varphi_2]_P, \min_{t \leq t'' \leq t'} \left( [w,t'' \models \varphi_1]_P \right) \right) \right)\\
		%		[w,t \models \varphi_1 \since \varphi_2]_F &=  \max_{t' \leq t} \left( \min \left( [w,t' \models \varphi_2]_F, \min_{t \leq t'' \leq t'} \left( [w,t'' \models \varphi_1]_F \right) \right) \right)\\
	\end{align*}
	
	LTL3:
	\begin{align*}
		[w,t \models \varphi]_3 &= \begin{cases}
			\top \text{, if for all }  w' : \mathbb{P} \times [0, \infty) \to \B  \text{ we have } \tau(w) \tau(w') \in \mathcal{L}(\varphi)\\
			\bot \text{, if for all }  w' : \mathbb{P} \times [0, \infty) \to \B  \text{ we have } \tau(w) \tau(w') \notin \mathcal{L}(\varphi)\\
			? \text{, otherwise}\\
		\end{cases}
	\end{align*}
	
	LTL5:
	\begin{align*}
		[w,t \models \varphi]_{5} &= \begin{cases}
			\top \text{, if } [w,t \models \varphi]_3 = \top \\
			\top_c \text{, if } [w,t \models \varphi]_3 = ? \text{ and } [w,t \models \varphi]_F = \top \\
			? \text{, if } [w,t \models \varphi]_3 = ? \text{ and } [w,t \models \varphi]_F = ? \\
			\bot_c \text{, if } [w,t \models \varphi]_3 = ? \text{ and } [w,t \models \varphi]_F = \bot \\
			\bot \text{, if } [w,t \models \varphi]_3 = \bot \\
		\end{cases} \\
	\end{align*}	
	

%	\rednote[inline]{An alternative: Define the semantics with the global view in mind by only considering if the propositions involved in a subformula belong to the same process or not. This requires the assumption that the propositions in disjunctions are ordered by the processes they belong to (or some other canonical form).}
	
	\subsubsection{Derived operators}
	Defined as usual.
	
	\subsection*{Offline Algorithm} \rednote{TODO: Check for bugs.}
	The algorithm consists of a preprocessing steps followed by the monitoring algorithm.
	First, we update the structure of the input formula to enable the local evaluation of the subformulas whose propositions belong to the same process.
	Then, the monitoring step evaluates the updated subformulas incrementally on the refined input signal.
	
	Let $\varphi$ be an LTL5 formula and $w : \mathbb{P} \times [0,d) \to \B$ be an $n$-dimensional finite-length signal.
	
	\begin{enumerate}
		\item Preprocessing: Restructuring of the subformulas and the input signal
		\begin{enumerate}[label=\arabic*.]
			\item Enumerate the subformulas of $\varphi$ such that each formula has an enumeration number smaller than the numbers of all its subformulas. Let $\varphi_0, \ldots, \varphi_m$ be such an enumeration. Note that $\varphi_0 = \varphi$.
			\item For each process $P$, let $m_P$ be the least number such that all the propositions that occur in $\varphi_{m_P}$ belong to $P$. If $\varphi_{m_P}$ is not an atomic proposition, define a fresh proposition on process $P$. Let $\mathbb{P}'$ be the set of propositions extended with the fresh ones.
			\item Let $\varphi'$ be the formula obtained by replacing the appropriate subformulas of $\varphi$ with the corresponding fresh propositions. Let $\varphi_0', \ldots, \varphi_{m'}'$ be its subformulas satisfying the enumeration invariant given above. Note that $\varphi_0'$  and $\varphi$ are semantically equivalent. %some redundancy to handle here
			\item Define a new signal $w' : \mathbb{P}' \times [0,d) \to \B$ as follows: for all $p \in \mathbb{P}'$ and $t \in [0,d)$, let \\
			- $w'(p,t) = w(p,t)$ if $p \in \mathbb{P}$, and\\
			- $w'(p,t) = [w,t \models \varphi_{m_{\pi(p)}}]_{\pi(p)}$ otherwise.
		\end{enumerate}
		
%		\item Preprocessing: $(\varepsilon,\delta)$-retiming of the restructured input signal
%		\begin{enumerate}[label=\arabic*.]
%			\item Let $p \in \mathbb{P}'$ be a proposition and $t_1, \ldots, t_k$ be the increasing sequence of timestamps corresponding to the events on $p$ of the signal $w'$. Let $T = \{t_1, \ldots, t_k\}$. Let $\ell_p$ be an empty list.
%			\item Let $\ell$ be an empty list. Add the lowest timestamp $t$ in $T$ to $\ell$ and remove it from $T$. While the greatest timestamp in $\ell$ and the lowest timestamp $t'$ in $T$ are less than $\delta + 2\varepsilon$ apart, add $t'$ to $\ell$ and remove it from $T$.
%			\item Compute the regions of uncertainty for the proposition $p$ under the assumption of bounded variability:\\ for $i = 1 .. |\ell|$ let $N_i = \left[\max_{j \in \{1, i\}} \ell[j] - \varepsilon + (i-j)\delta, \min_{i \leq j \leq |\ell|} \ell[j] + \varepsilon - (j-i)\delta\right]$ %\rednote{assuming $\ell[1] \geq \varepsilon$ for now to make it look cleaner}
%			and add $N_i$ to $\ell_p$.
%			\item Repeat (2)-(3) until $T$ is empty. Store the list $\ell_p$ of intervals in memory.
%			\item Repeat (1)-(4) for all $p \in \mathbb{P}'$.
%			\item Define a new signal $w'' : \mathbb{P}' \times [0,d) \to \B \cup \{\?\}$ as follows: for all $p \in \mathbb{P}'$ and $t \in [0,d)$, let \\
%			- $w''(p,t) = \?$ if $t \in N$ for some $N \in \ell_p$, \\
%			- $w''(p,t) = w'(p,0)$ if $t \notin N$ for any $N \in \ell_p$, and there is an even number of intervals in $\ell_p$ before $t$, and \\
%			- $w''(p,t) = \overline{w'(p,0)}$ if $t \notin N$ for any $N \in \ell_p$, and there is an odd number of intervals in $\ell_p$ before $t$.
%		\end{enumerate}
		
		\item Monitoring
		\begin{enumerate}[label=\arabic*.]
			\item Let $t_1, \ldots, t_k$ be an increasing sequence of timestamps corresponding to the events of the propositions in $w'$. Let $t_0 = 0$.
			\item For $i = k .. 0$, for $j = m' .. 0$, compute $[w', t_i \models \varphi_j]_5$. Note that the FLTL interpretations are evaluated from the perspective of the monitor process.
			\item Output $[w',0 \models \varphi_0']_5$.
		\end{enumerate}
	\end{enumerate}
	
	
	\subsection*{Online Algorithm [idea]}
	\textbf{Initial idea:}
	The monitor outputs verdicts with a delay of $h = \delta + 2\varepsilon$.
	At each event, the interval borders of the corresponding proposition needs to be recomputed.
	However, if there are no new events in the horizon of $h$, then we can stop updating the intervals we have computed so far because whatever happens later will not affect the previous events.
	This helps us provide an upper bound on the time and space requirements because in any chain of uncertainty intervals of length $L$, there are at most $\floor{\frac{L}{\delta}} + 1$ events.
	
	\vspace{1em}
	\noindent
	\textbf{A counter-example:} Consider a one-dimensional infinite-length signal $w$ with the following sequence of timestamps for its events
	$$ t, t + \delta - \varepsilon, t + 2\delta - \frac{3\varepsilon}{2}, t + 3\delta - \frac{7\varepsilon}{4}, \ldots$$
	Once the first event is observed, its uncertainty upper bound is $t + \varepsilon$.
	After the second event, it becomes $t$, then $t - \frac{\varepsilon}{2}$, and $t - \frac{3\varepsilon}{4}$ and so on.
	Regardless of how distant a future event is from the first event, it affects its uncertainty upper bound.
	This sequence of timestamps is indeed valid because for all $k \geq 1$
	$$ \floor{\frac{(k-1)\delta + 2\varepsilon - (2\varepsilon \cdot \sum_{i=1}^{k}2^{-k})}{\delta}} + 1 \geq k $$

	
	\newpage
	\bibliographystyle{splncs04}
	\bibliography{main}
\end{document}