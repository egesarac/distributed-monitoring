%\vspace{-4mm}
\section{Introduction} 
\label{sec:introduction}

%\ege{I added \textbackslash usepackage[compact]{titlesec} -- is this allowed?}

\todo{describe the setting, say we call it distributed monitoring}

\todo{make it explicit why we care about a faster solution to this problem}

\emph{Distributed systems} are networks of independent agents that work together to achieve a 
common objective.
%
Distributed systems are everywhere around us and come in many different forms.
%
For example, cloud computing uses distribution of resources and services over the internet to offer 
to their users a scalable infrastructure with transparent on-demand access to computing power and 
storage. 
%
Swarms of drones represent another family of distributed systems where individual drones 
collaborate to accomplish tasks like surveillance, search and rescue, or package delivery.
%
While each drone operates independently, it also communicates and coordinates with others to 
successfully achieve their common objectives.
%
The individual agents in a distributed system typically do not share a global clock.
%
To coordinate actions across multiple agents, clock synchronization is often needed.
%
While perfect clock synchronization is impractical due to network latency and node failures, 
algorithms such as the Network Time Protocol (NTP) allow agents to maintain a \emph{bounded 
skew} between the synchronized clocks.
%
In that case, we say that a distributed system has \emph{partial synchrony}. 

Formal verification of distributed system is a notoriously hard problem, due to the combinatorial 
explosion of all possible interleavings in the behaviors collected from individual agents.
%
 \emph{Runtime verification (RV)} provides a more pragmatic approach, in which a monitor observes 
 a behavior of a distributed system and checks its correctness against a formal specification.
 %
%\subsection{Motivation}
There has been emerging interest in designing RV techniques for distributed systems in the past few 
years.
%
Remotely related to the problem of distributed RV under partial synchrony are distributed RV in the 
fully {\em synchronous}~\cite{ef20,cf16,bf16} and {\em 
asynchronous}~\cite{cgnm13,mg05,og07,mb15,g20,bfrrt22} settings as well as  benchmarking 
tools~\cite{aafi21} for assessing monitoring overhead.
%
The problem of distributed RV under partial synchrony assumption has been studied for Linear 
 Temporal Logic (LTL)~\cite{GangulyMB20} and Signal Temporal Logic (STL)~\cite{MomtazAB23} 
 specification languages.
 %
 The proposed solutions use Satisfiability-Modulo-Theory (SMT) solving to provide sound and 
 complete distributed monitoring procedures.
 %
 Although distributed RV monitors consume only a single distributed behavior at a time, this behavior 
 can nevertheless have an excessive number of possible interleavings. 
  %
 Put it another way, although RV deals only with verification of a single execution at run time, it is 
 still prone to evaluating an explosion of combinations.
 %
 Hence, the exact distributed monitors from the literature can still suffer from significant 
 computational overhead.
 %
 This phenomenon has been observed even under partial synchrony~\cite{GangulyMB20,gmb24}.
  
%  \subsection{Contributions}

To mitigate this issue, we present a novel approach for \emph{approximate} RV of STL 
specifications under partial synchrony.
%
In essence, we abstract away potential interleavings in distributed behaviors in a conservative 
manner, resulting in an effective overapproximation of global behaviors.
%
This abstraction simplifies the representation of distributed behaviors into a set of Boolean 
expressions, taking into account regions of uncertainty created by clock skews.
%
We define monitoring operations required to evaluate temporal specifications over such expressions, 
which result in monitoring verdicts on overapproximate behaviors.
%
There is typically an inevitable tradeoff between {\em accuracy} and {\em speedup} in approximate 
solutions, and  RV is not an exception.
%
That is, gains in the monitoring speedup may result in reduced accuracy.
%
For some applications, reduced accuracy may not be acceptable.
%
Therefore, we propose a methodology that combines our approximate monitors with their exact 
counterparts, with the aim to benefit from the enhanced monitoring efficiency without sacrificing 
precision.
%
Approximate monitoring is also valuable in the sequential setting, with applications including
monitoring with state estimation \cite{StollerBSGHSZ11,BartocciG13},
quantitative monitoring and its resource-precision tradeoffs \cite{HenzingerS21,HenzingerMS22,HenzingerMS23},
and various other uses \cite{AlechinaDL14,AcetoAFIL21}.



We implemented our approach in a prototype tool and performed thorough evaluations on both 
synthetic and real-world case studies (mutual separation in swarm of drones and a water distribution 
system).
%
We first demonstrated that in synthetic experiments, our approximate monitors consistently achieve 
speedups of between 3 and 5 orders of magnitudes compared to the exact SMT-based solution.
%
We empirically characterized the classes of specifications and behaviors for which our approximate 
monitoring approach achieves good precision.
%
In our case studies, we observe up to 5 orders of magnitude speedup (although in some cases the 
improvement is not as much).
%
We finally show that combining exact and approximate distributed RV yields significant efficiency gains on average without sacrificing precision, even when approximate monitors had low accuracy.

%\paragraph{Organization.}
%%\noindent\textit{Organization.}
%Section~\ref{sec:preliminaries} presents the preliminary concepts.
%%
%Section~\ref{sec:semantics} introduces overapproximate semantics of STL while 
%Sections~\ref{sec:approach} and~\ref{sec:algorithm} describe our approach in approximating traces 
%and the associated monitoring algorithm, respectively.
%%
%We evaluate our approach in Section~\ref{sec:experiments}. Finally, we conclude in 
%Section~\ref{sec:conclusion}.