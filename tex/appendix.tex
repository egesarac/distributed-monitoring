\section*{Appendix}
\subsubsection{Proof of \cref{cl:stlsound}.}
\begin{proof}
	Let $\varphi$ be an STL formula and $(S,{\hb})$ be a distributed signal.
	Assume $[(S,{\hb}) \models \varphi]_+ = \top$.
	We want to show that $[(S,{\hb}) \models \varphi] = \top$.
	Expanding the definition of $[(S,{\hb}) \models \varphi]_+ = \top$, we have $w \models \varphi$ for all $w \in \tr^+(S,{\hb})$.
	By \cref{cl:trsound}, we have $\tr(S,{\hb}) \subseteq \tr^+(S,{\hb})$.
	Then, it holds that $w \models \varphi$ for all $w \in \tr(S,{\hb})$.
	Therefore, $[(S,{\hb}) \models \varphi] = \top$ by definition.
	The case of $[(S,{\hb}) \models \varphi]_+ = \bot$ follows from the same arguments.
\end{proof}

\subsubsection{Proof of \cref{cl:trsound}.}
\begin{proof}
	Let $(S,{\hb})$ be a distributed signal where $S = (x_1, \ldots, x_n)$.
	Let $w = (y_1, \ldots, y_n) \in \tr(S,{\hb})$ be a trace.
	We want to show that $w \in \tr^+(S,{\hb})$.
	First, let us recall the definition of $\tr^+$.
	\[ \tr^+(S,{\hb}) = \{ (x_1', \ldots, x_n') \st \text{$x_i'$ is consistent with $x_i$ for all $1 \leq i \leq n$}\} \]
	
	Let $1 \leq i \leq n$ be arbitrary.
	To show that $y_i$ is consistent with $x_i$, we need to show that $y_i$ is $I$-consistent with $x_i$ for all $I \in G_S$.
	Let $I = [t_0, s)$ be an arbitrary segment in $G_S$, let $(t_1, y_i(t_1)), \ldots, (t_\ell, y_i(t_\ell))$ be the edges of $y_i$ in segment $I$ with $t_j < t_{j+1}$ for all $1 \leq j < \ell$.
	To show that $y_i$ is $I$-consistent with $x_i$, we need to show that the expression $y_i(t_0) \cdot y_i(t_1) \cdot \ldots \cdot y_i(t_\ell)$ belongs to $\gamma(x_i,I)$.
	We sketch the proof idea below.
	
	Note that $w$ can be seen as a trace obtained through an $\varepsilon$-retiming of $S$ (see \cite[Section 4.2]{MomtazAB23}).
	Then, the timestamp $t$ of any edge of $x_i$ is mapped to some clock value in the range $(\theta_{\text{lo}}(t), \theta_{\text{hi}}(t))$.
	In particular, $|t - c^{-1}_i(t)| < \varepsilon$ for all $t \in \{t_0, t_1, \ldots, t_\ell\}$, where $c^{-1}_i(t)$ is the local clock value of $x_i$ that is mapped to $t$.
	
	Since $y_i$ has $\ell$ edges in $I$, it holds that $x_i$ has at least $\ell$ edges in $(t_0 - \varepsilon, s + \varepsilon)$.
	Since $I$ is a segment in $G_S$, there are $\ell$ of these that are consecutive such that the intersection of their uncertainty regions contain $(t_0,s)$, i.e., $(t_0,s) \subseteq \bigcap_{1 \leq j \leq \ell} (\theta_{\text{lo}}(t_j'), \theta_{\text{hi}}(t_j'))$ where $t_j' = c^{-1}_i(t_j)$ is the corresponding timestamp in $x_i$ for all $0 \leq j \leq \ell$.
	In particular, note that $y_i(t_j) = x_i(t_j')$ for all $0 \leq j \leq \ell$.
		
	Now, notice that, by definition, $\gamma(x_i, I)$ takes into account every edge of $x_i$ whose uncertainty region has a nonempty intersection with $I$, and preserves their order.
	Let $V_j$ be the set of value expressions capturing how $I$ relates with the uncertainty intervals of the edge $(t_j', x_i(t_j'))$ for all $1 \leq j \leq \ell$ (as defined in \cref{eq:valexprset}).
	Then, $\destutter(\{x_i(t_0')\} \cdot V_1 \cdot \ldots \cdot V_\ell) \subseteq \gamma(x_i, I)$.
	One can verify that for all $1 \leq j \leq \ell$, either $x_i(t_j')$ or $x_i(t_{j-1}') \cdot x_i(t_j')$ belongs to $V_j$.
	This allows us to choose a value expression $v_j$ from each $V_j$ such that $\destutter(\{x_i(t_0')\} \cdot v_1 \cdot \ldots \cdot v_\ell) = x_i(t_0') \cdot x_i(t_1') \cdot \ldots \cdot x_i(t_\ell')$, which concludes the proof. 

	Note that if there are more edges of $x_i$ with a timestamp smaller than $t_0'$ or larger than $t_\ell'$ whose uncertainty intervals intersect with $I$, then the corresponding set of value expressions is obtained either by prefixing or suffixing.
	In either case, we can choose $\epsilon$ from these sets for concatenation with the remaining edges' value expressions and obtain the desired result.
\end{proof}

\subsubsection{Proof of \cref{cl:algo}.}

\alert{TODO: Fix}

Let $w = (x_1, \ldots, x_n)$ be a (synchronous) trace and $E_w = \{t_1, \ldots, t_m\}$ be the set of timestamps for its edges with $t_j < t_{j+1}$ for all $1 \leq j < m$.
We define $\pi(w) = (v_1, \ldots, v_n)$ where $v_i = x_i(0) \cdot x_i(t_1) \cdot \ldots \cdot x_i(t_m)$ for all $1 \leq i \leq n$.
Let $(S,{\hb})$ be a distributed signal.
We define
\[ \pi(\tr^+(S,{\hb})) = \{ \pi(w) \st w \in \tr^+(S,{\hb}) \}. \]
Note that for each $u \in \pi(\tr^+(S,{\hb}))$ we have $u = \destutter(u)$.

Let $k \in \N$ and recall the function $\stutter_k : \Sigma^* \to \Sigma^*$.
For $L \subseteq \Sigma^*$, we define $\stutter_k(L) = \bigcup_{u \in L} \stutter_k(u)$.

Let $(S,{\hb})$ be a distributed signal such that $S = (x_1, \ldots, x_n)$ and $G_S = \{I_1, \ldots\ I_k\}$ where the segments with lower indices start (and end) before those with higher indices.
Moreover, we define 
\[ \Omega(S,{\hb}) = \{ \destutter(u) \st u \in \Gamma(S,{\hb}) \} \]
where
\[ \Gamma(S,{\hb}) = ( \stutter_3(\gamma(x_1, I_1)) \cdot \ldots \cdot \stutter_3(\gamma(x_1, I_k)), \ldots,  \stutter_3(\gamma(x_n, I_1)) \cdot \ldots \cdot \stutter_3(\gamma(x_n, I_k)) ). \]
Note that $\stutter_3$ only serves to capture the interleavings caused by the asynchronity within the segments.

\begin{lemma} \label{cl:eq}
	For every distributed signal $(S,{\hb})$, we have $\pi(\tr^+(S,{\hb})) = \Omega(S,{\hb})$.
\end{lemma}
\begin{proof}
	We briefly sketch each direction.
	The key observation is that both sets of value expressions are essentially defined through $\gamma$.

	Let $u \in \pi(\tr^+(S,{\hb}))$.
	By definition, $u$ is obtained from a trace $(x_1', \ldots, x_n')$ such that, for each $1 \leq i \leq n$, the signal $x_i'$ is $I$-consistent with $x_i$ for all $I \in G_S$.
	Namely, for each $1 \leq i \leq n$ and $I \in G_S$, the value expression obtained by concatenating the initial value of $x_i'$ in $I$ and the values of its edges in $I$ belongs to $\gamma(x_i, I)$.
	Then, one can verify that $u \in \Omega(S,{\hb})$.

	Let $u \in \Omega(S,{\hb})$.
	Recall that $u$ is obtained by choosing, for each $1 \leq i \leq n$ and $I \in G_S$, a value expression $v_{i,I} \in \gamma(x_i, I)$.
	By definition, there exists $(x_1', \ldots, x_n') \in \tr^+(S,{\hb})$ such that, for each $1 \leq i \leq n$ and $I \in G_S$, the behavior of $x_i'$ in the interval $I$ is given by the value expression $v_{i,I}$.
	Then, one can verify that $u \in \pi(\tr^+(S,{\hb}))$.
\end{proof}

We now proceed to prove \cref{cl:algo}.

\begin{proof}[\cref{cl:algo}]
	The proof goes by induction on the structure of $\varphi$.
	Let $(S,{\hb})$ be a distributed signal and $\varphi$ an STL formula.
	We sketch each case below.
%	Let $G_S$ be the canonical segmentation of $(S,{\hb})$ and let $I \in G_S$ be the first segment.
	
	\noindent\textbf{Atomic propositions.}
	For the base case, let $\varphi = p$ for some $p \in \AP$.
	Note that $\llbracket (S, {\hb}), I \models p \rrbracket$ expands to $\gamma(x_p, I)$ where $x_p$ is the signal encoding $p$.
	Then, it is easy to see that for every $u \in \Omega(S,{\hb})$ the component corresponding to $x_p$ starts with the value $\first(\gamma(x_p, I))$ (which is a singleton).
	Moreover, by \cref{cl:eq}, the same holds for $\pi(\tr^+(S,{\hb}))$, thus we conclude $[(S,{\hb}) \models p]_+ = \top$ (resp. $\bot$) iff $\first(\llbracket (S, {\hb}) \models p \rrbracket) = \{1\}$ (resp. $\{0\}$).

	\noindent\textbf{Negation.}
	Let $\varphi = \lnot \psi$ for some formula $\psi$ such that $[(S,{\hb}) \models \psi]_+$ and $\first(\llbracket (S, {\hb}) \models \psi \rrbracket)$ agree.
	If $[(S,{\hb}) \models \psi]_+ = \bot$ (resp. $\top$, $\,?$) and $\first(\llbracket (S, {\hb}) \models \psi \rrbracket) = \{0\}$ (resp. $\{1\}$, $\{0,1\}$), then, thanks to their inductive definitions, it is easy to see that $[(S,{\hb}) \models \lnot \psi]_+ = \top$ (resp. $\bot$, $\,?$) and $\first(\llbracket (S, {\hb}) \models \lnot \psi \rrbracket) = \{1\}$ (resp. $\{0\}$, $\{1,0\}$).

	\noindent\textbf{Conjunction.}
	Let $\varphi = \psi_1 \land \psi_2$ for some formulas $\psi_1, \psi_2$ such that $[(S,{\hb}) \models \psi_i]_+$ and $\first(\llbracket (S, {\hb}) \models \psi_i \rrbracket)$ agree for each $i \in \{1,2\}$.
	Suppose $[(S,{\hb}) \models \psi_i]_+ = \top$ and $\first(\llbracket (S, {\hb}) \models \psi_i \rrbracket) = \{1\}$ agree for each $i \in \{1,2\}$.
	Then,  $[(S,{\hb}) \models \psi_1 \land \psi_2]_+ = \top$ because $tr^+(S,{\hb})$ being contained in the languages of $\psi_1$ and $\psi_2$ implies that it is also contained in their intersection, and $\first(\llbracket (S, {\hb}) \models \psi_i \rrbracket) = \{1\}$ because the conjunction of any asynchronous product where both components start with 1 also starts with 1.
	The remaining cases follow from similar arguments.
		
	\noindent\textbf{Untimed until.}
	Let $\varphi = \psi_1 \land \psi_2$ for some formulas $\psi_1, \psi_2$ such that $[(S,{\hb}) \models \psi_i]_+$ and $\first(\llbracket (S, {\hb}) \models \psi_i \rrbracket)$ agree for each $i \in \{1,2\}$.
	
	
	
	\noindent\textbf{Timed until.}
	\TODO
\end{proof}

\newpage

