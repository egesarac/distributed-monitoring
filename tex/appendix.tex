\section*{Appendix}

\begin{proof}[Proof of \cref{cl:stlsound}]
	Let $\varphi$ be an STL formula and $(S,{\hb})$ be a distributed signal.
	Assume $[(S,{\hb}) \models \varphi]_+ = \top$.
	We want to show that $[(S,{\hb}) \models \varphi] = \top$.
	Expanding the definition of $[(S,{\hb}) \models \varphi]_+ = \top$, we have $w \models \varphi$ for all $w \in \tr^+(S,{\hb})$.
	By \cref{cl:trsound}, we have $\tr(S,{\hb}) \subseteq \tr^+(S,{\hb})$.
	Then, it holds that $w \models \varphi$ for all $w \in \tr(S,{\hb})$.
	Therefore, $[(S,{\hb}) \models \varphi] = \top$ by definition.
	The case of $[(S,{\hb}) \models \varphi]_+ = \bot$ follows from the same arguments.
\end{proof}

\begin{proof}[Proof of \cref{cl:trsound}]
	Let $(S,{\hb})$ be a distributed signal where $S = (x_1, \ldots, x_n)$.
	Let $w = (y_1, \ldots, y_n) \in \tr(S,{\hb})$ be a trace.
	We want to show that $w \in \tr^+(S,{\hb})$.
	First, recall the following:
	\[ \tr^+(S,{\hb}) = \{ (x_1', \ldots, x_n') \st \text{$x_i'$ is consistent with $x_i$ for all $1 \leq i \leq n$}\} \]
	To achieve our goal, we need to show that $x_i$ and $y_i$ are consistent for all $1 \leq i \leq n$.
	Let $1 \leq i \leq n$ be arbitrary, and let $I = [t_0, s)$ be a segment in $G_S$.
	Moreover, let $(t_1, y_i(t_1)), \ldots, (t_\ell, y_i(t_\ell))$ be the edges of $y_i$ in segment $I$ with $t_j < t_{j+1}$ for all $1 \leq j < \ell$.
	We want to show that the expression $y_i(t_0) \cdot y_i(t_1) \cdot \ldots \cdot y_i(t_\ell)$ belongs to $\gamma(x_i,I)$.
	We sketch the proof below.
	
	Note that $w$ can be seen as a trace obtained through an $\varepsilon$-retiming of $S$ (see \cite[Section 4.2]{MomtazAB23}).
	It is then clear that the timestamps of the events (and thus edges) of $x_i$ are mapped to clock values that are less than $\varepsilon$ away.
	In particular, $|t - c^{-1}_i(t)| < \varepsilon$ for all $t \in \{t_0, t_1, \ldots, t_\ell\}$, where $c^{-1}_i(t)$ is the local clock value of $x_i$ that is mapped to $t$.
	Since $y_i$ has $\ell$ edges in $I$, it holds that $x_i$ has at least $\ell$ edges in $[t_0 - \varepsilon, s + \varepsilon)$.
	Since $I$ is a segment in $G_S$, there are $\ell$ of these that are consecutive such that the intersection of their uncertainty regions contain $(t_0,s)$, i.e., $(t_0,s) \subseteq \bigcap_{1 \leq j \leq \ell} (\theta_{\text{lo}}(t_j'), \theta_{\text{hi}}(t_j'))$ where $t_j' = c^{-1}_i(t_j)$ is the corresponding timestamp in $x_i$ for all $0 \leq j \leq \ell$.
	In particular, note that $y_i(t_j) = x_i(t_j')$ for all $0 \leq j \leq \ell$.
		
	Now, notice that, by definition, $\gamma(x_i, I)$ takes into account every edge of $x_i$ whose uncertainty region has a nonempty intersection with $I$.
	Moreover, thanks to concatenation in its definition, it contains a value expression in which all $\ell$ edges occur and their order is preserved.
	We conclude that $x_i(t_0') \cdot x_i(t_1') \cdot \ldots \cdot x_i(t_\ell') \in \gamma(x_i, I)$.
\end{proof}

\begin{proof}[Proof of \cref{cl:algo}]
	\TODO
\end{proof}