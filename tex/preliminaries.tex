\section{Preliminaries} \label{sec:preliminaries}

%We define boolean domain $\B = \{ \bot, \top \}$ as the set of boolean truth values, where $\bot < \top$ and they 
%complement each other, i.e., $\overline{\bot} = \top$ and $\overline{\top} = \bot$.
%
We denote by $\B = \{ \top, \bot \}$ the set of Booleans, $\R$ the set of reals, $\R_{\geq 0}$ the set of nonnegative reals, and $\R_{> 0}$ the 
set of positive reals.
%
An interval $I \subseteq \R$ of reals with the end points $a < b$ has length $|b-a|$.

Let $\Sigma$ be a finite {\em alphabet}.
%
We denote by $\Sigma^*$ the set of finite words over $\Sigma$ and by $\epsilon$ the empty word.
%
For $u \in \Sigma^*$, we respectively write $\pfx(u)$ and $\sfx(u)$ for the sets of prefixes 
and suffixes of $u$.
%
We also let $\infx(u) = \{v \in \Sigma^* \st \exists x,y \in \Sigma^* : u = xvy\}$.
%
For a nonempty word $u \in \Sigma^*$ and $1 \leq i \leq |u|$, we denote by $u[i]$ the $i$th letter of 
$u$, by $u[..i]$ the prefix of $u$ of length $i$, and by $u[i..]$ the suffix of $u$ of length $|u| - i + 1$. 
%
Given $u \in \Sigma^*$ and $\ell \geq 1$, we denote by $u^\ell$ the word obtained by concatenating $u$ by itself $\ell - 1$ times.
Moreover, given $L \subseteq \Sigma^*$, we define $\first(L) = \{ u[0] \st u \in L\}$.
For sets $L_1, L_2 \subseteq \Sigma^*$ of words, we let $L_1 \cdot L_2 = \{u \cdot v \st u \in L_1, v \in L_2\}$.
For tuples $(u_1, \ldots, u_m)$ and $(v_1, \ldots, v_m)$ of words, we let $(u_1, \ldots, u_m) \cdot (v_1, \ldots, v_m) = (u_1 v_1, \ldots, u_m v_m)$.

We define the function $\destutter : \Sigma^* \to \Sigma^*$ inductively.
For all $\sigma \in \Sigma \cup \{\epsilon\}$, let $\destutter(\sigma) = \sigma$.
For all $u \in \Sigma^*$ such that $u = \sigma_1 \sigma_2 v$ for some $\sigma_1,\sigma_2 \in 
\Sigma$ and $v \in \Sigma^*$, we define it as follows.

\small
\begin{equation*}
	\destutter(u) =
	\begin{cases}
		\destutter(\sigma_2 v) & \text{if } \sigma_1 = \sigma_2 \\
		\sigma_1 \cdot \destutter(\sigma_2 v) & \text{otherwise}
	\end{cases}
\end{equation*}
\normalsize
%For all $\sigma \in \Sigma \cup \{\epsilon\}$, let $\destutter(\sigma) = \sigma$.
%%
%For all $u \in \Sigma^*$ such that $u = \sigma_1 \sigma_2 v$ for some $\sigma_1,\sigma_2 \in 
%\Sigma$ and $v \in \Sigma^*$, let (i) $\destutter(u) = \destutter(\sigma_2 v)$ if $\sigma_1 = 
%\sigma_2$, and (ii) $\destutter(u) = \sigma_1 \cdot \destutter(\sigma_2 v)$ otherwise.
%%
For a set $L \subseteq \Sigma^*$ of finite words, we define $\destutter(L) = 
\{\destutter(u) \st u \in L\}$.
%
To extend $\destutter$ to tuples of words, for all $\sigma \in \Sigma \cup \{\epsilon\}$  let $\destutter(\sigma, \ldots, \sigma) = (\sigma, \ldots, \sigma)$.
Given a tuple $(u_1, \ldots, u_m) = (\sigma_{1,1} \sigma_{1,2} v_1, \ldots, \sigma_{m,1} \sigma_{m,2} v_m)$ of finite words of the same length, we define $\destutter(u_1, \ldots, u_m)$ as expected:

\small
\begin{equation*}
	\destutter(u_1, \ldots, u_m) =
	\begin{cases}
		\destutter(\sigma_{1,2} v_1, \ldots, \sigma_{m,2} v_m) \text{ if $\sigma_{i,1} = \sigma_{i,2}$ for all $1 \leq i \leq m$} \\
		(\sigma_{1,1}, \ldots, \sigma_{m,1}) \cdot \destutter(\sigma_{1,2} v_1, \ldots, \sigma_{m,2} v_m) \text{ otherwise}
	\end{cases}
\end{equation*}
\normalsize
%Given a tuple $(u_1, \ldots, u_m) = (\sigma_{1,1} \sigma_{1,2} v_1, \ldots, \sigma_{m,1} \sigma_{m,2} v_m)$ of finite words of the same length, we define $\destutter(u_1, \ldots, u_m)$ as expected: (i) $\destutter(u_1, \ldots, u_m) = \destutter(\sigma_{1,2} v_1, \ldots, \sigma_{m,2} v_m)$ if $\sigma_{i,1} = 
%\sigma_{i,2}$ for all $1 \leq i \leq m$, and (ii) $\destutter(u_1, \ldots, u_m) = (\sigma_{1,1}, \ldots, \sigma_{m,1}) \cdot \destutter(\sigma_{1,2} v_1, \ldots, \sigma_{m,2} v_m)$ otherwise.

Moreover, given an integer $k \geq 0$, we define $\stutter_k : \Sigma^* \to \Sigma^*$ such that $\stutter_k(u) = \{v \in \Sigma^* \st |v| = k \land \destutter(v) = \destutter(u)\}$ if $k \geq |\destutter(u)|$, and $\stutter_k(u) = \emptyset$ otherwise.

\subsubsection{Signal Temporal Logic (STL) \cite{MalerN13}.} \label{sec:stl}

Let $A,B \subset \R$.
%
A function $f : A \to B$ is
\emph{right-continuous} iff $\lim_{a \to c^+} f(a) = f(c)$ for all $c \in A$, and
%\emph{left-limited} iff $\lim_{a \to c^-} f(a) < \infty$ for all $c \in A$;
\emph{non-Zeno} iff for every bounded interval $I \subseteq A$ there are finitely many $a \in I$ such that $f$ is not continuous at $a$.
%
A \emph{signal} is a right-continuous, non-Zeno, piecewise-constant function $x : [0,d) \to \R$ where $d \in \R_{> 0}$ is the duration of $x$ and $[0,d)$ is its temporal domain.
Let $x : [0,d) \to \R$ be a signal.
An \emph{event} of $x$ is a pair $(t, x(t))$ where $t \in [0,d)$.
An \emph{edge} of $x$ is an event $(t, x(t))$ such that $\lim_{s \to t^-} x(s) \neq \lim_{s \to t^+} x(s)$.
In particular, an edge is \emph{rising} if $\lim_{s \to t^-} x(s) < \lim_{s \to t^+} x(s)$, and it is \emph{falling} otherwise.
A signal $x : [0,d) \to \R$ can be represented finitely by its initial value and edges: if $x$ has $m$ edges, then $x = (t_0, v_0) (t_1, v_1) \ldots (t_m, v_m)$ such that $t_0 = 0$, $t_{i-1} < t_i$, and $(t_i, v_i)$ is an edge of $x$ for all $1 \leq i \leq m$.

Let $\AP$ be a set of \emph{atomic propositions}.
The syntax is given by the following grammar where $p \in \AP$ and $I \subseteq \R_{\geq 0}$ is an interval.
\[ \varphi :=  p ~|~ \lnot \varphi ~|~ \varphi \land \varphi ~|~ \varphi \until_I \varphi \]

A \emph{trace} $w = (x_1, \ldots, x_n)$ is a finite vector of signals.
We express atomic propositions as functions of trace values at a time point $t$,
i.e., a proposition $p \in \AP$ over a trace $w = (x_1, \ldots, x_n)$ is defined as $f_p(x_1(t), \ldots, x_n(t)) > 0$ where $f_p : \R^n \to \R$ is a function.
%i.e., a proposition $p \in \AP$ over a trace $w = (x_1, \ldots, x_n)$ is defined as $f_p(x_1(t), \ldots, x_n(t)) \sim_p c_p$ where $f_p : \R^n \to \R$ is a function, $c_p \in \R$ is a constant, and ${\sim_p} \in \{{<}, {\leq}, {\geq}, {>}\}$.
Given intervals $I,J \subseteq \R_{\geq 0}$, we define $I \oplus J = \{i + j \st i \in I \land j \in J\}$, and we simply write $t$ for the singleton set $\{t\}$. 

Below, we recall the finite-trace qualitative semantics of STL defined over $\B$ \cite{MalerN13}.
Let $d \in \R_{> 0}$ and $w = (x_1, \ldots, x_n)$ with $x_i : [0,d) \to \R$ for all $1 \leq i \leq n$.
Let $\varphi_1, \varphi_2$ be STL formulas and let $t \in [0,d)$.

\small
\begin{align*}
	(w,t) \models p \iff & f_p(x_1(t), \ldots, x_n(t)) > 0 \\
	(w,t) \models \lnot \varphi_1 \iff & \overline{(w,t) \models \varphi_1} \\
	(w,t) \models \varphi_1 \land \varphi_2 \iff & (w,t) \models \varphi_1 \land (w,t) \models \varphi_2 \\
	(w,t) \models \varphi_1 \until_I \varphi_2 \iff & \exists t' \in (t \oplus I) \cap [0,d) :  \\
	& (w,t') \models \varphi_2 \land \forall t'' \in (t, t') : (w,t'') \models \varphi_1
\end{align*}
\normalsize

We simply write $w \models \varphi$ for $(w,0) \models \varphi$.
We additionally use the following standard abbreviations: 
$\false = p \land \lnot p$,
$\true = \lnot \false$,
$ \varphi_1 \lor \varphi_2 = \lnot (\lnot \varphi_1 \land \lnot \varphi_2)$,
$\LTLf_I \varphi = \true \until_I \varphi$, and
$\LTLg_I \varphi = \lnot \LTLf_I \lnot \varphi$.
Moreover, the untimed temporal operators are defined through their timed counterparts on the interval $[0,\infty)$, e.g., $\varphi_1 \until \varphi_2 = \varphi_1 \until_{[0,\infty)} \varphi_2$.
%Note that our interpretation of the untimed until operator is strict.
%The non-strict variant can be defined in terms of the strict one as follows: $\varphi_1 \underline{\until} \varphi_2 = \varphi_2 \lor (\varphi_1 \land (\varphi_1 \until \varphi_2))$.

%The semantics above is defined for infinite traces while a distributed signal has finite length.
%To bridge this gap, we take the standard extension to a 3-valued semantics.
%Given a finite-length trace $w$ and an STL formula $\varphi$, 
%we define $[w \models \varphi]_3 = \top$ iff $[w w' \models \varphi]_{\mathsf{STL}}$ for every infinite-length signal $w'$, define $[w \models \varphi]_3 = \bot$ iff $[w w' \models \lnot \varphi]_{\mathsf{STL}}$ for every infinite-length signal $w'$, and $[w \models \varphi]_3 = \?$ otherwise.


\subsubsection{Distributed Semantics of STL \cite{MomtazAB23}.} \label{sec:diststl}

We consider an asynchronous and loosely-coupled message-passing system of $n \geq 2$ reliable agents producing a set of signals $x_1, \ldots, x_n$, where for some $d \in \R_{> 0}$ we have $x_i : [0,d) \to \R$ for all $1 \leq i \leq n$.
%
The agents do not share memory or a global clock.
%
Only to formalize statements, we speak of a \emph{hypothetical} global clock and denote its value by $T$.
%
For local time values, we use the lowercase letters $t$ and $s$.

For a signal $x_i$, we denote by $V_i$ the set of its events, by $E_i^\uparrow$ the set of its rising edges, and by $E_i^\downarrow$ that of falling edges.
Moreover, we let $E_i = E_i^\uparrow \cup E_i^\downarrow$.
%
We represent the local clock of the $i$th agent as an increasing and divergent function $c_i : 
\R_{\geq 0} \to \R_{\geq 0}$ that maps a global time $T$ to a local time $c_i(T)$.

We assume that the system is \emph{partially synchronous}: the agents use a clock synchronization algorithm that guarantees a bounded clock skew with respect to the global clock, i.e., $|c_i(T) - c_j(T)| < \varepsilon$ for all $1 \leq i,j \leq N$ and $T \in \R_{\geq 0}$, where $\varepsilon \in \R_{> 0}$ is the maximum clock skew.

\begin{definition} \label{defn:hb}
	A \emph{distributed signal} is a pair $(S, {\hb})$, where $S = (x_1, \ldots, x_n)$ is a vector of 
	signals and ${\hb}$ is the happened-before relation between events in signals extended with the 
	partial synchrony assumption as follows.
	\begin{itemize}
		\item For every agent, the events of its signals are totally ordered, i.e., for all $1 \leq i \leq n$ and all $(t, x_i(t)), (t', x_i(t')) \in V_i$, if $t < t'$ then $(t, x_i(t)) \hb (t', x_i(t'))$.
		\item Every pair of events whose timestamps are at least $\varepsilon$ apart is totally ordered, i.e., for all $1 \leq i,j \leq n$ and all $(t, x_i(t)) \in V_i$ and $(t', x_j(t')) \in V_j$, if $t + \varepsilon \leq t'$ then $(t, x_i(t)) \hb (t', x_j(t'))$.
	\end{itemize}
\end{definition}

\begin{example}
	\alert{TODO: distributed signal, happened-before relation}
\end{example}

\begin{definition}
	Let $(S, {\hb})$ be a distributed signal of $n$ signals, and $V = \bigcup_{i = 1}^{n} V_i$ be the set of its events.
	A set $C \subseteq V$ is a \emph{consistent cut} iff for every event in $C$, all events that happened before  it also belong to $C$, i.e., for all $e, e' \in V$, if $e \in C$ and $e' \hb e$, then $e' \in C$.
\end{definition}

We denote by $\CC(T)$ the (infinite) set of consistent cuts at global time $T$.
Given a consistent cut $C$, its \emph{frontier} $\fr(C) \subseteq C$ is the set consisting of the last events in $C$ of each signal, i.e., $\fr(C) = \bigcup_{i = 1}^{n} \{ (t, x_i(t)) \in V_i \cap C \st \forall t' > t : (t', x_i(t')) \notin V_i \cap C \}$.

\begin{definition}
A \emph{consistent cut flow} is a function $\ccf : \R_{\geq 0} \to 2^V$ that maps a global clock value $T$ to the frontier of a consistent cut at time $T$, i.e., $\ccf(T) \in \{\fr(C) \st C \in \CC(T)\}$.
\end{definition}

For all $T,T' \in \R_{\geq 0}$ and $1 \leq i \leq n$, if $T < T'$, then for every pair of events $(c_i(T), x_i(c_i(T))) \in \ccf(T)$ and $(c_i(T'), x_i(c_i(T'))) \in \ccf(T')$ we have $(c_i(T), x_i(c_i(T))) \hb (c_i(T'), x_i(c_i(T')))$.
We denote by $\CCF(S,{\hb})$ the set of all consistent cut flows of the distributed signal $(S,{\hb})$.

\begin{example}
	\alert{TODO: consistent cut, frontier, consistent cut flow}
\end{example}

Observe that a consistent cut flow of a distributed signal induces a vector of synchronous signals which can be evaluated using the standard semantics described in \cref{sec:stl}.
Let $(S,{\hb})$ be a distributed signal of $n$ signals $x_1, \ldots, x_n$.
A consistent cut flow $\ccf \in \CCF(S,{\hb})$ yields a trace $w_{\ccf} = (x'_1, \ldots x'_n)$ on the temporal domain $[0,d)$ such that $(c_i(T), x_i(c_i(T))) \in \ccf(T)$ implies $x_i'(T) = x_i(c_i(T))$ for all $1 \leq i \leq n$ and $T \in [0, d)$.
The set of traces of $(S,{\hb})$ is given by $\tr(S,{\hb}) = \{ w_{\ccf} \st \ccf \in \CCF(S,{\hb})\}$.

We define the satisfaction of an STL formula $\varphi$ by a distributed signal $(S,{\hb})$ over a three-valued domain $\{\top, \bot, {?}\}$
If the set of synchronous traces $\tr(S,{\hb})$ defined by a distributed signal $(S,{\hb})$ is contained in the set of traces allowed by the formula $\varphi$, then $(S,{\hb})$ satisfies $\varphi$.
Similarly, if $\tr(S,{\hb})$ has an empty intersection with the set of traces $\varphi$ defines, then $(S,{\hb})$ violates $\varphi$.
Otherwise, the evaluation is inconclusive since some traces satisfy the property and some violate it.
Notice that we quantify universally over traces for both satisfaction and violation.

\small
\begin{equation*}
	[(S,{\hb}) \models \varphi] = 
	\begin{cases}
		\top & \text{ if } \forall w \in \tr(S,{\hb}) : w \models \varphi \\
		\bot & \text{ if } \forall w \in \tr(S,{\hb}) : w \models \lnot\varphi \\
		\,? & \text{ otherwise}
	\end{cases}
\end{equation*}
\normalsize

\section{Overapproximation of the STL Distributed Semantics}
To address the computational overhead in exact distributed monitoring, we define STL$^+$, a variant of STL whose syntax is the same as STL but semantics provide a sound approximation of the STL distributed semantics.
STL$^+$ is better adapted to distributed monitoring as it overapproximates the set of traces.
In particular, given a distributed signal $(S,{\hb})$, STL$^+$ considers an approximation 
$\tr^+(S,{\hb})$ of the set $\tr(S,{\hb})$ of synchronous traces where $ \tr(S,{\hb}) \subseteq \tr^+(S,{\hb})$.
A signal $(S,{\hb})$ satisfies (resp. violates) an STL$^+$ formula $\varphi$ iff all the traces in $\tr^+(S,{\hb})$ belong to the language of $\varphi$ (resp. $\lnot \varphi$).

\small
\begin{equation*}
	[(S,{\hb}) \models \varphi]_+ = 
	\begin{cases}
		\top & \text{ if } \forall w \in \tr^+(S,{\hb}) : w \models \varphi \\
		\bot & \text{ if } \forall w \in \tr^+(S,{\hb}) : w \models \lnot\varphi \\
		\,? & \text{ otherwise}
	\end{cases}
\end{equation*}
\normalsize

In \cref{sec:approach,sec:algorithm}, we respectively define $\tr^+$ and present an algorithm to compute the semantics of STL$^+$.
We finally prove the correctness of our approach.

\begin{theorem} \label{cl:stlsound}
	For every STL formula $\varphi$ and every distributed signal $(S,{\hb})$, if $[(S,{\hb}) \models \varphi]_+ = \top$ (resp. $\bot$) then $[(S,{\hb}) \models \varphi] = \top$ (resp. $\bot$).
\end{theorem}

Notice that both the distributed semantics of STL and the semantics of STL$^+$ quantify universally over the set of traces for the verdicts $\top$ and $\bot$.
Therefore, the \cref{cl:stlsound} holds for the verdicts $\top$ and $\bot$, but not for ${\,?}$.
