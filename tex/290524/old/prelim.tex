\section{Preliminaries}

We define $\B = \{ \bot, \top \}$ as the set of boolean truth values, where $\bot < \top$ and they 
complement each other, i.e., $\overline{\bot} = \top$ and $\overline{\top} = \bot$.
%
We denote by $\R$ the set of reals, $\R_{\geq 0}$ the set of nonnegative reals, and $\R_{> 0}$ the 
set of positive reals.
%
An interval $I \subseteq \R$ of reals with the end points $a < b$ has length $|b-a|$.

Let $\Sigma$ be a finite {\em alphabet}.
%
We denote by $\Sigma^*$ the set of finite words over $\Sigma$ and by $\epsilon$ the empty word.
%
For $u \in \Sigma^*$, we respectively write $\pfx(u)$ and $\sfx(u)$ for the sets of nonempty prefixes 
and suffixes of $u$.
%
We also let $\infx(u) = \{v \in \Sigma^* \st \exists x,y \in \Sigma^* : u = xvy \land v \neq \epsilon\}$.
%
For a nonempty word $u \in \Sigma^*$ and $1 \leq i \leq |u|$, we denote by $u[i]$ the $i$th letter of 
$u$, by $u[..i]$ the prefix of $u$ of length $i$, and by $u[i..]$ the suffix of $u$ of length $|u| - i + 1$. 
%
Given $u \in \Sigma^*$ and $\ell \geq 1$, we denote by $u^\ell$ the word obtained by concatenating $u$ by itself $\ell - 1$ times.
Moreover, given $L \subseteq \Sigma^*$, we define $\first(L) = \{ u[0] \st u \in L\}$.

We define the function $\destutter : \Sigma^* \to \Sigma^*$ inductively as follows.
For all $\sigma \in \Sigma \cup \{\epsilon\}$, let $\destutter(\sigma) = \sigma$.
%
For all $u \in \Sigma^*$ such that $u = \sigma_1 \sigma_2 v$ for some $\sigma_1,\sigma_2 \in 
\Sigma$ and $v \in \Sigma^*$, let (i) $\destutter(u) = \destutter(\sigma_2 v)$ if $\sigma_1 = 
\sigma_2$, and (ii) $\destutter(u) = \sigma_1 \cdot \destutter(\sigma_2 v)$ otherwise.
%
By extension, for a set $L \subseteq \Sigma^*$ of finite words, we write $\destutter(L) = 
\{\destutter(u) \st u \in L\}$.
%
Given a tuple $(u_1, \ldots, u_m)$ of finite words of the same length, we write $\destutter(u_1, \ldots, u_m)$ for the extension of $\destutter$ defined as expected: requiring the equality condition in (i) to hold for all the words in the tuple.


Moreover, given an integer $k \geq 0$, we define $\stutter_k : \Sigma^* \to \Sigma^*$ such that $\stutter_k(u) = \{v \in \Sigma^* \st |v| = k \land \destutter(v) = \destutter(u)\}$ if $k \geq |\destutter(u)|$, and $\stutter_k(u) = \emptyset$ otherwise.

\subsection{Signal Model}

Let $A,B \subset \R$.
%
A function $f : A \to B$ is
\emph{right-continuous} iff $\lim_{a \to c^+} f(a) = f(c)$ for all $c \in A$, and
%\emph{left-limited} iff $\lim_{a \to c^-} f(a) < \infty$ for all $c \in A$;
\emph{non-Zeno} iff for every bounded interval $I \subseteq A$ there are finitely many $a \in I$ such that $f$ is not continuous at $a$.


\begin{definition}
	A \emph{signal} is a right-continuous, non-Zeno, piecewise-constant function $x : [0,d) \to \R$ where $d \in \R_{> 0}$ is the duration of $x$ and $[0,d)$ is its temporal domain.
\end{definition}

We consider an asynchronous and loosely-coupled message-passing system of $N \geq 2$ reliable 
agents, denoted $A_1, \ldots, A_N$.
%
The agents produce a set of $n \geq 2$ signals $x_1, \ldots, x_n$, where for some $d \in \R_{> 0}$ 
we have $x_i : [0,d) \to \R$ for all $1 \leq i \leq n$.
%
The function $\pi : \{x_1, \ldots, x_n\} \to \{A_1, \ldots, A_N\}$ maps each signal to the agent it is 
produced by. \borzoo{Check later if this is really needed.}
%
The agents do not share memory or a global clock.
%
Only to formalize statements, we speak of a {\em hypothetical} global clock and denote its value by 
$T$.
%
For local time values, we use the lowercase letters $t$ and $s$.

We represent the local clock of the agent $A_i$ as an increasing and divergent function $c_i : 
\R_{\geq 0} \to \R_{\geq 0}$ that maps a global time $T$ to a local time $c_i(T)$ of $A_i$.
%
We denote by $c_i^{-1}$ the inverse of the local clock function $c_i$.
%
We assume that the system is \emph{partially synchronous}: the agents use a clock synchronization algorithm that guarantees a bounded clock skew with respect to the global clock, i.e., $|c_i(T) - T| < \varepsilon$ for all $1 \leq i \leq N$ and $T \in \R_{\geq 0}$, where $\varepsilon \in \R_{> 0}$ is the maximum clock skew.
\borzoo{Another way is $|c(i) - c(j)| < \epsilon$} \anik{$|c_i(T) - c_j(T)| < \epsilon$ for all $1 \leq i,j \leq N$ and $T \in \R_{\geq 0}$}



An \emph{event} of a given signal $x_i$ is a pair $(t, x_i(t))$, where $t$ is a local clock value.
We denote by $V_i$ the set of events of signal $x_i$.
%
An \emph{edge} of $x_i$ is an event $(t, x_i(t))$ such that $\lim_{s \to t^-} x_i(s) \neq \lim_{s \to t^+} x_i(s)$.
In particular, it is a \emph{rising} edge if $\lim_{s \to t^-} x_i(s) < \lim_{s \to t^+} x_i(s)$, and a \emph{falling} edge otherwise.
We respectively denote by $E_i^\uparrow$ and $E_i^\downarrow$ the sets of rising and falling edges of $x_i$, and we let $E_i = E_i^\uparrow \cup E_i^\downarrow$.

We assume that signals have \emph{bounded variability} with respect to the global clock: for each 
signal $x_i$ and every pair $(t, x_i(t)), (t', x_i(t')) \in E_i$ of edges, we have $|c_j^{-1}(t) - 
c_j^{-1}(t')| \geq \delta$  \anik{for all $1 \leq j \leq N$}, where $A_j = \pi(x_i)$ is the agent that produces $x_i$ and $\delta \in 
\R_{\geq 0}$ is the bounded variability constant. \borzoo{Do we really need the agent here?}

\begin{definition} \label{defn:hb}
	A \emph{distributed signal} is a pair $(S, {\hb})$, where $S = (x_1, \ldots, x_n)$ is a vector of 
	signals and ${\hb}$ is the happened-before relation between events in signals extended with the 
	partial synchrony assumption as follows.
	\begin{itemize}
		\item For every agent, the events of its signals are totally ordered, i.e., for all $1 \leq i,j \leq n$ \anik{Use either `n' or `N' consistently} 
		with $\pi(x_i) = \pi(x_j)$ and all $(t, x_i(t)) \in V_i$ and $(t', x_j(t')) \in V_j$, if $t < t'$ then $(t, 
		x_i(t)) \hb (t', x_j(t'))$.
		\borzoo{I do think $\pi$ is unnecessary. We can assume each agent produces a distinct signal.}
		\item Every pair of events whose timestamps are at least $2 \varepsilon$ apart is totally ordered, i.e., for all $1 \leq i,j \leq n$ and all $(t, x_i(t)) \in V_i$ and $(t', x_j(t')) \in V_j$, if $t + 2\varepsilon \leq t'$ then $(t, x_i(t)) \hb (t', x_j(t'))$.
	\end{itemize}
\end{definition}

\begin{example}
	\TODO: distributed signal, happened-before relation
\end{example}

\begin{definition}
	Let $(S, {\hb})$ be a distributed signal of $n$ signals, and $V = \bigcup_{i = 1}^{n} V_i$ be the set of its events.
	A set $C \subseteq V$ is a \emph{consistent cut} iff for every event in $C$, all events that happened before  it also belong to $C$, i.e., for all $e, e' \in V$, if $e \in C$ and $e' \hb e$, then $e' \in C$.
\end{definition}

We denote by $\CC(T)$ the (infinite) set of consistent cuts at global time $T$.
Given a consistent cut $C$, its \emph{frontier} $\fr(C) \subseteq C$ is the set consisting of the last events in $C$ of each signal, i.e., $\fr(C) = \bigcup_{i = 1}^{n} \{ (t, x_i(t)) \in V_i \cap C \st \forall t' > t : (t', x_i(t')) \notin V_i \cap C \}$.

\begin{definition}
A \emph{consistent cut flow} is a function $\ccf : \R_{\geq 0} \to 2^V$ that 
maps a global clock value $T$ to the frontier of a consistent cut at time 
$T$, i.e., $\ccf(T) \in \{\fr(C) \st C \in \CC(T)\}$.
\end{definition}


For all $T,T' \in \R_{\geq 0}$ and $1 \leq i \leq n$ where $A_j = \pi(x_i)$, if 
$T < T'$, then for every pair of events $(c_j(T), x_i(c_j(T))) \in \ccf(T)$ and 
$(c_j(T'), x_i(c_j(T'))) \in \ccf(T')$ we have $(c_j(T), x_i(c_j(T))) \hb (c_j(T'), 
x_i(c_j(T')))$.
We denote by $\CCF(S,{\hb})$ the set of all consistent cut flows of the distributed signal $(S,{\hb})$.

\begin{example}
	\TODO: consistent cut, frontier, consistent cut flow
\end{example}


\subsection{Signal Temporal Logic} \label{sec:stl}
Let $\AP$ be a set of atomic propositions.
The syntax is given by the following grammar where $p \in \AP$ and $I \subseteq \R_{\geq 0}$ is an interval.
$$ \varphi :=  p ~|~ \lnot \varphi ~|~ \varphi \land \varphi ~|~ \varphi \until_I \varphi$$

A \emph{trace} is a vector of synchronous signals of finite duration.
We express atomic propositions as functions of trace values at a time point 
$T$, i.e., a proposition $p \in \AP$ over a trace $w = (x_1, \ldots, x_n)$ is 
defined as $f_p(x_1(T), \ldots, x_n(T)) > 0$ where $f_p : \R^n \to \R$ is a 
function.
Note that given a trace over a temporal domain $[0,D)$ and intervals $I,J \subseteq \R_{\geq 0}$, we write $I \oplus J = \{i + j \st i \in I \land j \in J\}$.
We use the shorthand notation $T$ for the singleton set $\{T\}$. 
\dejan{We are talking both about $x_i(t)$ and $x_i(T)$ and these two things have a different meaning. I do not know whether we should say something more about it. $x_i(t)$ is the trace observation on the agent $A_i$ and its local clock, while $x_i(T)$ is the (unknown) real value of the signal wrt the global time.}
\ege{I try to use $x_i(T)$ only when the signals are synchronous (i.e., for talking about traces). When there is asynchrony, I write $x_i(c_j(T)))$ where $A_j$ is the agent producing $x_i$.}

Below we recall the finite-trace qualitative semantics of STL defined over $\B$.
Let $D \in \R_{> 0}$ and $w = (x_1, \ldots, x_n)$ with $x_i : [0,D) \to \R$ for all $1 \leq i \leq n$.
Let $\varphi_1, \varphi_2$ be STL formulas and let $T \in [0,D)$.

\small
\begin{align*}
	[w,T \models p]_{\mathsf{STL}} &\iff & f_p(x_1(T), \ldots, x_n(T)) > 0 \\
	[w,T \models \lnot \varphi_1]_{\mathsf{STL}} &\iff & \overline{[w,T \models \varphi_1]_{\mathsf{STL}}} \\
	[w,T \models \varphi_1 \land \varphi_2]_{\mathsf{STL}} &\iff & [w,T \models \varphi_1]_{\mathsf{STL}} \land [w,T \models \varphi_2]_{\mathsf{STL}} \\
	[w,T \models \varphi_1 \until_I \varphi_2]_{\mathsf{STL}} &\iff & \exists T' \in (T \oplus I) \cap [0,D) : [w,T' \models \varphi_2]_{\mathsf{STL}} \land \\
	& & \forall T'' \in (T, T') : [w,T'' \models \varphi_1]_{\mathsf{STL}} \\
%	[w,t \models \LTLf_I \varphi_1] &\iff \exists t' \in t \oplus I : [w,t' \models \varphi_1] \\
%	[w,t \models \LTLg_I \varphi_1] &\iff \forall t' \in t \oplus I : [w,t' \models \varphi_1] \\
\end{align*}
\normalsize

We simply write $[w \models \varphi]$ for $[w,0 \models \varphi]$.
We additionally use the following standard abbreviations: 
$\false = p \land \lnot p$,
$\true = \lnot \false$,
$ \varphi_1 \lor \varphi_2 = \lnot (\lnot \varphi_1 \land \lnot \varphi_2)$,
$\LTLf_I \varphi = \true \until_I \varphi$, and
$\LTLg_I \varphi = \lnot \LTLf_I \lnot \varphi$.
Moreover, the untimed temporal operators are defined through their timed counterparts on the interval $(0,\infty)$, e.g., $\varphi_1 \until \varphi_2 = \varphi_1 \until_{(0,\infty)} \varphi_2$.
Note that our interpretation of the untimed until operator is strict.
The non-strict variant can be defined in terms of the strict one as follows: $\varphi_1 \underline{\until} \varphi_2 = \varphi_2 \lor (\varphi_1 \land (\varphi_1 \until \varphi_2))$.

%The semantics above is defined for infinite traces while a distributed signal has finite length.
%To bridge this gap, we take the standard extension to a 3-valued semantics.
%Given a finite-length trace $w$ and an STL formula $\varphi$, 
%we define $[w \models \varphi]_3 = \top$ iff $[w w' \models \varphi]_{\mathsf{STL}}$ for every infinite-length signal $w'$, define $[w \models \varphi]_3 = \bot$ iff $[w w' \models \lnot \varphi]_{\mathsf{STL}}$ for every infinite-length signal $w'$, and $[w \models \varphi]_3 = \?$ otherwise.