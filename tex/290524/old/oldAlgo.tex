\subsubsection*{Assumptions}
\begin{enumerate}
	\item Every atomic proposition is uniquely mapped to a process, which transmit the changes in their truth values to a central monitor. The initial values of each atomic proposition is known by the monitor.
	%		\item The monitor runs on a process which does not own any propositions. \rednote{to confirm}
	\item The monitor runs on the same process as the reference clock.
	\item There exists $\varepsilon > 0$ such that the local clocks of the processes are guaranteed to be less than $\varepsilon$ apart from the reference clock at any given point. The value of the least upper bound on $\varepsilon$ is known by the monitor. %There exists $\varepsilon > 0$ such that the local clocks of every pair of processes are guaranteed to be less than $\varepsilon$ apart at any given point. The value of the least upper bound on $\varepsilon$ is known by the monitor. 
	\item There exists $\delta > 0$ such that the values of the atomic propositions stay the same after each change for at least $\delta$ time units with respect to the reference clock. The value of the greatest lower bound on $\delta$ is known by the monitor.
	\item The processes do not fail, the communication channels are FIFO and lossless.
	\item For online monitoring: The events may arrive out of order to the monitor. \rednote{Not taken into account yet.}
\end{enumerate}

\subsection*{FLTL for Distributed Monitoring under Partial Synchrony}
Let $\mathcal{P} = \{P_1, \ldots, P_m\}$ be a set of $m \geq 2$ processes and $\mathbb{P} = \{p_1, \ldots, p_n\}$ be a set of atomic propositions.
The function $\pi : \mathbb{P} \to \mathcal{P}$ maps each proposition to the process it belongs to.
Let $\B = \{\bot, \top\}$ be a boolean value domain with $\bot < \top$, and let $d$ be a positive real number.
A boolean signal $w : \mathbb{P} \times [0,d] \to \B$ maps each atomic proposition and time value to a boolean truth value.
For a given a signal $w : \mathbb{P} \times [0,d] \to \B$, an event $e = (p, t, l, b) \in \mathbb{P} \times [0,d] \times \N \times \B$ records a change in the value of $p$ at (local) time $t$ where $l \in \N$ stands for the value of the logical clock of the process $\pi(p)$ and $b$ stands for the value of $p$ after the event.
Given an event $e = (p, t, l, b)$, we respectively denote by $e.p$, $e.t$, $e.l$, and $e.b$ the corresponding proposition, (local) time value, logical clock value, and truth value.

%	Note that every boolean signal $w$ can be represented in a discrete manner by identifying all the rising and falling edges.
%	Given a signal $w : \mathbb{P} \times [0,d] \to \B$ and a time value $0 \leq t < d$, let $\tau(w,t)$ be the discrete representation of the signal corresponding to the projection of $w$ on the temporal domain $[t, d)$.
%	When $t = 0$, we simply write $\tau(w)$. 
%	Given an LTL formula $\psi$, we denote by $\mathcal{L}(\psi)$ the language of $\psi$ which consists of all infinite-length signals $w$ whose discrete representation $\tau(w)$ satisfies $\psi$.

%	We define a 5-valued logic LTL5 to account for the clock skew.
%	The truth domain of LTL5 consists of 5 values with $\bot < \bot_c < \? < \top_c < \top$, where $\overline{\bot} = \top$, $\overline{\bot_c} = \top_c$, and $\overline{\?} = \?$.
%	To define LTL5, we first incorporate the value $\?$ in the standard logics FLTL and LTL3.

\subsubsection*{Syntax}
Defined by
$ \varphi := \texttt{true} \,|\, p \,|\, \lnot \varphi \,|\, \varphi \lor \varphi \,|\, \varphi \until \varphi $
where $p \in \mathbb{P}$.

\subsubsection*{Semantics}
Let $w : \mathbb{P} \times [0,d] \to \B$ be a boolean signal of finite length, $t \in [0,d]$ be a time value, $\varepsilon$ be the least upper bound on the clock skew, $\pi : \mathbb{P} \to \mathcal{P}$ a mapping from propositions to processes, $p \in \mathbb{P}$ be a proposition, and $P \in \mathcal{P}$ be a process.

Given $w : \mathbb{P} \times [0,d] \to \B$ and $p \in \mathbb{P}$, let $E_{w,p} = \{e_1, \ldots, e_m\}$ be the set of events of $p$ in $w$ ordered by their timestamps.
We define the (partial) functions $\rho_{\text{lo}}$ and $\rho_{\text{hi}}$ as follows.

$$ \rho_{\text{lo}}(w, p, e_i) = \max\{0, \max_{j \in \{1, i\}} e_j.t - \varepsilon - (j-i)\delta\} $$

$$ \rho_{\text{hi}}(w, p, e_i) = \min\{d, \min_{i \leq j \leq k} e_j.t + \varepsilon - (j-i)\delta\} $$
where $k = \max \{i \leq k' \leq m \st \forall i \leq j < k' : e_{j + 1}.t - e_j.t < \delta + 2\varepsilon\}$.
When $w$ and $p$ are clear from the context, we simply write $\rho_{\text{lo}}(e_i)$ and $\rho_{\text{hi}}(e_i)$.
%$\rho_{\text{hi}}(w, p, t^p_{j+1}) - (t^p_j + \varepsilon) < \delta$ this could be useful in the offline case

The interpretation of a formula depends on which process we observe the propositions from.
Note that each process knows the initial values of the propositions.

\vspace{1em}
FLTL:
\begin{align*}
	[w,t \models \texttt{true}]_P &= \top \\
	[w,t \models p]_P &= \begin{cases}
		? &\text{if } \pi(p) \not = P \land \exists e \in E_{w,p} : t \in (\rho_{\text{lo}}(e), \rho_{\text{hi}}(e)) \\
		e_i.b &\text{if } \pi(p) \not = P \land \forall e \in E_{w,p} : t \notin (\rho_{\text{lo}}(e), \rho_{\text{hi}}(e)) \\
		&\text{where } i = \max\{j \leq |E_{w,p}| : \rho_{\text{hi}}(e_j) \leq t \} \\
		%\alert{\rho_{\text{hi}}(e_i) \leq t \leq \rho_{\text{lo}}(e_{i+1})} \\
		w(p,t) &\text{if } \pi(p) = P \\
	\end{cases} \\
	[w,t \models \lnot \varphi]_P &= \overline{[w,t \models \varphi]_P}\\
	[w,t \models \varphi_1 \lor \varphi_2]_P &= \max([w,t \models \varphi_1]_P, [w,t \models \varphi_2]_P)\\
	[w,t \models \varphi_1 \until \varphi_2]_P &= \max_{t \leq t' \leq d} \left( \min \left( [w,t' \models \varphi_2]_P, \min_{t \leq t'' \leq t'} \left( [w,t'' \models \varphi_1]_P \right) \right) \right)\\
	%		[w,t \models \varphi_1 \since \varphi_2]_F &=  \max_{t' \leq t} \left( \min \left( [w,t' \models \varphi_2]_F, \min_{t \leq t'' \leq t'} \left( [w,t'' \models \varphi_1]_F \right) \right) \right)\\
\end{align*}

\rednote{TODO: Formalize.}

Let $w$ be a signal, $P$ be a process, and $M \neq P$ be the monitor process.
A valid retiming $r$ from $P$ to $M$ maps, for all $p \in \mathbb{P}$ with $\pi(p) = P$, every $e \in E_{w,p}$ to a timestamp $t \in (\rho_{\text{lo}}(e), \rho_{\text{hi}}(e))$ while preserving the order of events with respect to $P$.

\begin{claim}
	Let $w$ be a signal over $n$ processes, $\varphi$ be a propositional FLTL formula (without the until operator), and $M$ be the monitor process.
	We have
	\begin{enumerate}
		\item $[w,t \models \varphi]_M = \top$ iff for every set of $n-1$ valid retimings (one from each process to $M$) the retimed signal $w'$ evaluates to $\top$ at time $t$ under the standard FLTL semantics (without uncertainty).
		\item $[w,t \models \varphi]_M = \bot$ iff for every set of $n-1$ valid retimings (one from each process to $M$) the retimed signal $w'$ evaluates to $\bot$ at time $t$ under the standard FLTL semantics (without uncertainty).
		\item $[w,t \models \varphi]_M = \?$ otherwise.
	\end{enumerate}
\end{claim}
\begin{proof}[idea]
	Assume $\varphi$ is in disjunctive normal form.
	
	(1): Suppose $[w,t \models \varphi]_M = \top$.
	Thanks to the semantics of our logic, there is a clause in $\varphi$ for which, at time $t$, none of the propositions corresponding to its literals belong to a region of uncertainty and the literals evaluate to $\top$.
	Since a retiming only moves events within regions of uncertainty, the values of these literals do not change under any retiming.
	Then, the formula still evaluates to $\top$ at time $t$. 
	The other direction is similar.
	
	(2): Same arguments.
	
	(3): By elimination.
	
\end{proof}

%	\begin{definition}
	%		An \emph{$\varepsilon$-retiming} is an increasing function $r : \R_+ \to \R_+$ such that $|t - r(t)| < \varepsilon$ for all $t \in \R_+$. \rednote{need one retiming per process}
	%		\alert{Given a signal $w : \mathbb{P} \times [0,d] \to \B$ and its happened-before relation $\rightsquigarrow$, an $\varepsilon$-retiming (i) \emph{respects $\rightsquigarrow$} iff \ldots}, and (ii) \emph{respects $\delta$} iff the signal $w' : \mathbb{P} \times [0,r(d)) \to \B$ with $w'(p,r(t)) = w(p, t)$ ...
	%		An $\varepsilon$-retiming is \emph{valid} iff it respects both $\rightsquigarrow$ and $\delta$.
	%	\end{definition}


%	\rednote[inline]{An alternative: Define the semantics with the global view in mind by only considering if the propositions involved in a subformula belong to the same process or not. This requires the assumption that the propositions in disjunctions are ordered by the processes they belong to (or some other canonical form).}

\subsubsection{Derived operators}
Defined as usual.

\subsection*{Offline Algorithm} \rednote{TODO: Check for bugs.}
The algorithm consists of a preprocessing steps followed by the monitoring algorithm.
First, we update the structure of the input formula to enable the local evaluation of the subformulas whose propositions belong to the same process.
Then, the monitoring step evaluates the updated subformulas incrementally on the refined input signal.


Let $\varphi$ be an LTL5 formula and $w : \mathbb{P} \times [0,d] \to \B$ be an $n$-dimensional finite-length signal.

\begin{enumerate}
	\item Preprocessing: Restructuring of the subformulas and the input signal
	\begin{enumerate}[label=\arabic*.]
		\item Enumerate the subformulas of $\varphi$ such that each formula has an enumeration number smaller than the numbers of all its subformulas. Let $\varphi_0, \ldots, \varphi_m$ be such an enumeration. Note that $\varphi_0 = \varphi$.
		%			\item For each process $P$, let $m_P$ be the least number such that all the propositions that occur in $\varphi_{m_P}$ belong to $P$. If $\varphi_{m_P}$ is not an atomic proposition, define a fresh proposition on process $P$. Let $\mathbb{P}'$ be the set of propositions extended with the fresh ones. \rednote{TODO: a process may have more than one subformula we need to treat}
		\item For each process $P$, let $F_P$ be the set of subformulas of $\varphi$ such that for every  $\psi \in F_P$ we have (i) all the propositions that occur in $\psi$ belong to $P$, and (ii) in every subformula of $\varphi$ of which $\psi$ is a subformula, a proposition that does not belong to $P$ occurs. For each $\psi \in F_P$, if $\psi$ is not an atomic proposition, define a fresh proposition on process $P$. Let $\gamma$ be a bijection from the fresh propositions to the corresponding subformulas, and $\mathbb{P}'$ be the set of propositions extended with the fresh ones.
		\item Let $\varphi'$ be the formula obtained by replacing the appropriate subformulas of $\varphi$ with the corresponding fresh propositions. Let $\varphi_0', \ldots, \varphi_{m'}'$ be its subformulas satisfying the enumeration invariant given above. Note that $\varphi_0'$  and $\varphi$ are semantically equivalent. %some redundancy to handle here
		\item Define a new signal $w' : \mathbb{P}' \times [0,d] \to \B$ as follows: for all $p \in \mathbb{P}'$ and $t \in [0,d]$, let \\
		- $w'(p,t) = w(p,t)$ if $p \in \mathbb{P}$, and\\
		- $w'(p,t) = [w,t \models \gamma(p)]_{\pi(p)}$ otherwise.
	\end{enumerate}
	
	%		\item Preprocessing: $(\varepsilon,\delta)$-retiming of the restructured input signal
	%		\begin{enumerate}[label=\arabic*.]
		%			\item Let $p \in \mathbb{P}'$ be a proposition and $t_1, \ldots, t_k$ be the increasing sequence of timestamps corresponding to the events on $p$ of the signal $w'$. Let $T = \{t_1, \ldots, t_k\}$. Let $\ell_p$ be an empty list.
		%			\item Let $\ell$ be an empty list. Add the lowest timestamp $t$ in $T$ to $\ell$ and remove it from $T$. While the greatest timestamp in $\ell$ and the lowest timestamp $t'$ in $T$ are less than $\delta + 2\varepsilon$ apart, add $t'$ to $\ell$ and remove it from $T$.
		%			\item Compute the regions of uncertainty for the proposition $p$ under the assumption of bounded variability:\\ for $i = 1 .. |\ell|$ let $N_i = \left[\max_{j \in \{1, i\}} \ell[j] - \varepsilon + (i-j)\delta, \min_{i \leq j \leq |\ell|} \ell[j] + \varepsilon - (j-i)\delta\right]$ %\rednote{assuming $\ell[1] \geq \varepsilon$ for now to make it look cleaner}
		%			and add $N_i$ to $\ell_p$.
		%			\item Repeat (2)-(3) until $T$ is empty. Store the list $\ell_p$ of intervals in memory.
		%			\item Repeat (1)-(4) for all $p \in \mathbb{P}'$.
		%			\item Define a new signal $w'' : \mathbb{P}' \times [0,d] \to \B \cup \{\?\}$ as follows: for all $p \in \mathbb{P}'$ and $t \in [0,d]$, let \\
		%			- $w''(p,t) = \?$ if $t \in N$ for some $N \in \ell_p$, \\
		%			- $w''(p,t) = w'(p,0)$ if $t \notin N$ for any $N \in \ell_p$, and there is an even number of intervals in $\ell_p$ before $t$, and \\
		%			- $w''(p,t) = \overline{w'(p,0)}$ if $t \notin N$ for any $N \in \ell_p$, and there is an odd number of intervals in $\ell_p$ before $t$.
		%		\end{enumerate}
	
	\item Monitoring
	\begin{enumerate}[label=\arabic*.]
		\item Let $M$ be the process on which the monitor is running. Let $t_1, \ldots, t_k$ be an increasing sequence of timestamps corresponding to the events in $w'$. Let $t_0 = 0$.
		\item For $i = k .. 0$, for $j = m' .. 0$, compute $[w', t_i \models \varphi_j]_{M,5}$.
		\item Output $[w',0 \models \varphi_0']_{M,5}$.
	\end{enumerate}
\end{enumerate}


\subsection*{Online Algorithm}
\textbf{Initial idea:}
The monitor outputs verdicts with a delay of $h = \delta + 2\varepsilon$.
At each event, the interval borders of the corresponding proposition needs to be recomputed.
However, if there are no new events in the horizon of $h$, then we can stop updating the intervals we have computed so far because whatever happens later will not affect the previous events.
This helps us provide an upper bound on the time and space requirements because in any chain of uncertainty intervals of length $L$, there are at most $\floor{\frac{L}{\delta}} + 1$ events.

\vspace{1em}
\noindent
\textbf{A counter-example:} Consider a one-dimensional infinite-length signal $w$ with the following sequence of timestamps for its events

$$ t \hspace{2em} t + \delta - \varepsilon \hspace{2em} t + 2\delta - \frac{3\varepsilon}{2} \hspace{2em} t + 3\delta - \frac{7\varepsilon}{4} \hspace{2em} \ldots $$

Once the first event is observed, its uncertainty upper bound is $t + \varepsilon$.
After the second event, it becomes $t$, then $t - \frac{\varepsilon}{2}$, and $t - \frac{3\varepsilon}{4}$ and so on.
The upper bound converges to $t - \varepsilon$ in the limit.
Regardless of how distant a future event is from the first event, it affects its uncertainty upper bound.
This sequence of timestamps is indeed valid because for all $k \geq 1$

$$ \floor{\frac{(k-1)\delta + 2\varepsilon - (2\varepsilon \cdot \sum_{i=1}^{k}2^{-i})}{\delta}} + 1 \geq k $$
\rednote{TODO: Is there a reasonable assumption that would prevent this and guarantee a fixed finite horizon that applies to all events?}

\vspace{1em}
\noindent
\textbf{A simple option:}
Assume bounded variability with respect to local clocks (instead of the monitor's clock) and $\delta > 2\varepsilon$.
Then, introducing a delay of $\varepsilon$ to monitor verdicts, online monitoring of past-time formulas requires constant time per event and constant space overall (both linear in the size of the input formula) similar to~\cite{Havelund2002}.
\rednote{TODO: What if we have communication delay $\leq \gamma$ here? Further delay the verdict by $\gamma$, treat the events from different processes within this interval as concurrent?}

\subsection*{Reversal-boundedness}
Instead of the bounded variability assumption above, assume the following:
\begin{itemize}
	\item For all signals there exists $k \in \N$ such that each proposition associated with the signal changes its value at most $k$ times in total.
\end{itemize}

%	Note:
%	In this setting, we can't do much to refine long segments of uncertainty.
%	This could be especially problematic when we have to deal with uncertainty and a binary operation.