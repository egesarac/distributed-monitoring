\section{Experimental Evaluation}

\TODO: the setting etc.

\subsection{Optimizations}

\subsubsection{Data structures}

\TODO: computing boolean value expressions fast (the box representation and the operations)

\subsubsection{Preprocessing}

Restructuring of the subformulas and the input signal to improve the algorithm's 
precision.

\begin{enumerate}[label=\arabic*.]
	\item
	Enumerate the subformulas of $\varphi$ such that each formula has an enumeration number smaller than the numbers of all its subformulas.
	Let $\varphi_1, \ldots, \varphi_m$ be such an enumeration.
	Note that $\varphi_1 = \varphi$.
	
	\item
	%		Recall that the propositions $\AP$ of $\varphi$ are functions of the values of $x_1, \ldots, x_n$ at a given time point $t$.
	For each agent $A_i$, let $P_i$ be the set of subformulas of $\varphi$ such that for every $\psi \in P_i$ we have
	(i) $\pi(x) = A_i$ for each signal $x$ that occurs in $\psi$,
	(ii) the temporal operators $\psi$ contains, if any, are untimed, and
	(iii) for each subformula $\psi'$ of $\varphi$ such that $\psi$ is a subformula of $\psi'$, the first two conditions are violated.
	For each $\psi \in P_i$, if $\psi$ is not an atomic proposition, define a fresh signal $y$ such that $\pi(y) = A_i$.
	Let $y_1, \ldots, y_M$ be the fresh signals and let $\mathsf{f} : \{y_1, \ldots, y_M\} \to \{\varphi_1, \ldots, \varphi_m\}$ be function that maps each fresh signal to the corresponding subformula of $\varphi$.
	Then, $y_j(t) = [w,t \models \mathsf{f}(y_j)]_{\mathsf{STL}}$ for all $1 \leq j \leq M$ and $0 \leq t < d$, where $w$ is the trace obtained from $S$ by assuming complete synchrony.
	
	%		For each $\psi \in P_i$, if $\psi$ is not an atomic proposition, define a fresh proposition.
	%		Let $\AP_{\textit{new}} = \{q_1, \ldots, q_M\}$ be the set of fresh propositions and let $\mathsf{f} : \AP_{\textit{new} }\to \{\varphi_1, \ldots, \varphi_m\}$ be function that maps each fresh proposition to the corresponding subformula of $\varphi$.
	
	%		and let $\AP' = \AP_{\textit{new}} \cup \AP$.
	%		Let $\mathsf{f} : \AP' \to \{\varphi_0, \ldots, \varphi_m\}$ be function that maps each fresh proposition in $\AP_{\textit{new}}$ to the corresponding subformula and each (nonfresh) proposition of $\varphi$  in $\AP$ to itself.
	
	\item
	For each fresh signal $y_j$, define a fresh proposition $q_j = (y_j > 0)$ and replace in $\varphi$ the subformula $\mathsf{f}(q_j)$ with $q_j$.
	%		For each fresh proposition $q_i \in \AP_{\textit{new}}$, replace in $\varphi$ the subformula $\mathsf{f}(q_i)$ with the expression $y_i > 0$ where $y_i$ is the satisfaction signal of the subformula $\mathsf{f}(q_i)$ \alert{with respect to the 2-valued finite-trace semantics of STL}. \rednote{the way it's defined can be taken this way too}
	Let $\varphi'$ be the obtained formula.
	Let $\varphi_1', \ldots, \varphi_{m'}'$ be the subformulas of $\varphi'$ satisfying the enumeration invariant given above.
	Note that $\varphi_1'$  and $\varphi$ are semantically equivalent. %some redundancy to handle here
	
	\item
	We define a new distributed signal appropriately extending $(S,{\hb})$ with the fresh propositions.
	Let $(S', {\hb}')$ be a distributed signal with $S' = (x_1, \ldots, x_n, y_1, \ldots, y_M)$ and ${\hb}'$ the smallest extension of ${\hb}$ from $S$ to $S'$ satisfying \Cref{defn:hb}. 
	%where $y_i$ is the satisfaction signal of the proposition $q_i \in \AP_{\textit{new}}$ for each $1 \leq i \leq L$, 
	
\end{enumerate}

\subsection{Results}

\TODO